\chapter{The Programming Language \textsl{Python}}
We have started our lecture with an introduction to set theory.  In my experience, the notions of
set theory are difficult to master for many students because the concepts introduced in set theory
are quite abstract.  Fortunately, there is a programming language that supports sets as a basic data type and
thus enables us to experiment with set theory.  This is the programming language
\href{https://en.wikipedia.org/wiki/Python_(programming_language)}{Python}, which is available at
\href{http://www.python.org}{\texttt{python.org}}. \index{python}
By programming in \textsl{Python}, students can get acquainted with set theory in a playful manner.
Furthermore, as many interesting problems have a straightforward solution as \textsl{Python} programs,
students can appreciate the usefulness of abstract notions from set theory by programming in \textsl{Python}.
According to 
\href{https://cacm.acm.org/blogs/blog-cacm/176450-python-is-now-the-most-popular-introductory-teaching-language-at-top-u-s-universities/fulltext}{Philip Guo},
8 of the top 10 US universities teach \textsl{Python} in their introductory computer science courses.

The easiest way to install python and its libraries is via \href{https://www.anaconda.com/download/}{Anaconda}.
\index{Anaconda}
On many computers, \textsl{Python} is already preinstalled.  Nevertheless, even on those systems it is easiest
to use the \blue{Anaconda} distribution.  The reason is that Anaconda make it very easy to use different
versions of python with different libraries.  In this lecture, we will be using the version 3.9 of \textsl{Python}.

\section{Starting the Interpreter}
My goal is to introduce \textsl{Python} via a number of rather simple examples.  I will present more
advanced features of \textsl{Python} in later sections, but this section is intended to provide a first
impression of the language.


\begin{figure}[!ht]
\centering
\begin{Verbatim}[ frame         = lines, 
                  framesep      = 0.3cm, 
                  firstnumber   = 1,
                  labelposition = bottomline,
                  numbers       = none,
                  numbersep     = -0.2cm,
                  xleftmargin   = 0.0cm,
                  xrightmargin  = 0.0cm,
                ]
    Python 3.9.0 | packaged by conda-forge | (default, Oct 14 2020, 22:56:29) 
    [Clang 10.0.1 ] on darwin
    Type "help", "copyright", "credits" or "license" for more information.
\end{Verbatim}
\vspace*{-0.3cm}
\caption{The \textsl{Python} welcome message.}
\label{fig:python}
\end{figure}

The language \textsl{Python} is an \blue{interpreted} language.  Hence, there is no need to \blue{compile} a
program.  Instead, \textsl{Python} programs can be executed via the interpreter.  The interpreter is started
by the command:\footnote{
  While I am usually in the habit of terminating every sentence with either a full stop, a question
  mark, or an exclamation mark, I refrain from doing so when the sentence ends in a \textsl{Python} command
  that is shown on a separate line.  The reason is that I want to avoid confusion as it can
  otherwise be hard to understand which part of the line is the command that has to be typed
  verbatim.
}
\\[0.2cm]
\hspace*{1.3cm}
\texttt{python}
\\[0.2cm]
After the interpreter is started, the user sees the output that is shown in Figure 
\ref{fig:python} on page \pageref{fig:python}.  The string
``\texttt{\symbol{62}\symbol{62}\symbol{62}}'' is the \blue{prompt}.  It signals that the interpreter is waiting for input.
If we input the string
\\[0.2cm]
\hspace*{1.3cm}
\texttt{1 + 2}
\\[0.2cm]
and press enter, we get the following output:
\begin{verbatim}
    3
    >>> 
\end{verbatim}
The interpreter has computed the sum $1+2$, returned the result, and prints another prompt waiting
for more input.  Formally, the command ``\texttt{1 + 2}''
is a \blue{script}.  Of course, this is a very small script as it consists only of a single expression.
The command
\\[0.2cm]
\hspace*{1.3cm}
\texttt{exit()}
\\[0.2cm]
terminates the interpreter.   The nice thing about \textsl{Python}  is that we do not need a command line to
execute \textsl{Python} scripts, since we even  can run \textsl{Python} in a
browser in so called \href{https://en.wikipedia.org/wiki/Project_Jupyter}{jupyter notebooks}.  If you have
installed \textsl{Python} by means of the \href{https://www.anaconda.com/download/}{Anaconda} distribution,
then you already have installed these notebooks.  The following subsection contains the \blue{jupyter notebook} 
\href{https://github.com/karlstroetmann/Logic/blob/master/Python/Introduction.ipynb}{\texttt{Introduction.ipynb}}.
You should download this notebook from my github page and try the examples on your own computer.  Of course,
for this to work you first have to install \href{https://jupyter.org}{jupyter}.
\index{jupyter notebook}

\section{\texorpdfstring{An Introduction to
\textsl{Python}}{An Introduction to Python}}\label{an-introduction-to-python}
\href{https://github.com/karlstroetmann/Logic/blob/master/Python/Introduction.ipynb}{This} \textsl{Python} notebook gives a short introduction to \textsl{Python}.
We will start with the basics but as the main goal of this introduction
is to show how \textsl{Python} supports \blue{sets} we will quickly move
to more advanced topics. In order to show off the features of
\textsl{Python} we will give some examples that are not fully explained at
the point where we introduce them. However, rest assured that they will
be explained eventually.

\subsection{Evaluating expressions}\label{evaluating-expressions}
As Python is an interactive language, expressions can be evaluated
directly. In a \blue{Jupyter} notebook we just have to type Ctrl-Enter
in the cell containing the expression. Instead of Ctrl-Enter we can also
use Shift-Enter.

\begin{Verbatim}[commandchars=\\\{\}]
{\color{incolor}In [{\color{incolor}1}]:} \PY{l+m+mi}{1} \PY{o}{+} \PY{l+m+mi}{2}
\end{Verbatim}

\begin{Verbatim}[commandchars=\\\{\}]
{\color{outcolor}Out[{\color{outcolor}1}]:} 3
\end{Verbatim}
            
In \textsl{Python}, the precision of integers is not bounded. Hence, the
following expression does not cause an overflow.

    \begin{Verbatim}[commandchars=\\\{\}]
{\color{incolor}In[{\color{incolor}2}]:}\PY{l+m+mi}{1}\PY{o}{*}\PY{l+m+mi}{2}\PY{o}{*}\PY{l+m+mi}{3}\PY{o}{*}\PY{l+m+mi}{4}\PY{o}{*}\PY{l+m+mi}{5}\PY{o}{*}\PY{l+m+mi}{6}\PY{o}{*}\PY{l+m+mi}{7}\PY{o}{*}\PY{l+m+mi}{8}\PY{o}{*}\PY{l+m+mi}{9}\PY{o}{*}\PY{l+m+mi}{10}\PY{o}{*}\PY{l+m+mi}{11}\PY{o}{*}\PY{l+m+mi}{12}\PY{o}{*}\PY{l+m+mi}{13}\PY{o}{*}\PY{l+m+mi}{14}\PY{o}{*}\PY{l+m+mi}{15}\PY{o}{*}\PY{l+m+mi}{16}\PY{o}{*}\PY{l+m+mi}{17}\PY{o}{*}\PY{l+m+mi}{18}\PY{o}{*}\PY{l+m+mi}{19}\PY{o}{*}\PY{l+m+mi}{20}\PY{o}{*}\PY{l+m+mi}{21}\PY{o}{*}\PY{l+m+mi}{22}\PY{o}{*}\PY{l+m+mi}{23}\PY{o}{*}\PY{l+m+mi}{24}\PY{o}{*}\PY{l+m+mi}{25}
\end{Verbatim}


\begin{Verbatim}[commandchars=\\\{\}]
{\color{outcolor}Out[{\color{outcolor}2}]:} 15511210043330985984000000
\end{Verbatim}
            
    The next \blue{cell} in this notebook shows how to compute the
\blue{factorial} of 1000, i.e. it shows how to compute the product
\[ 1000! = 1 * 2 * 3 * \cdots * 998 * 999 * 1000 \] It uses some
advanced features from \blue{functional programming} that will be
discussed at a later stage of this introduction.

    \begin{Verbatim}[commandchars=\\\{\}]
{\color{incolor}In [{\color{incolor}3}]:} \PY{k+kn}{import} \PY{n+nn}{functools} 
        
        \PY{n}{functools}\PY{o}{.}\PY{n}{reduce}\PY{p}{(}\PY{k}{lambda} \PY{n}{x}\PY{p}{,} \PY{n}{y}\PY{p}{:} \PY{p}{(}\PY{n}{x}\PY{o}{*}\PY{n}{y}\PY{p}{)}\PY{p}{,} \PY{n+nb}{range}\PY{p}{(}\PY{l+m+mi}{1}\PY{p}{,} \PY{l+m+mi}{1001}\PY{p}{)}\PY{p}{)}
\end{Verbatim}


\begin{Verbatim}[commandchars=\\\{\}]
{\color{outcolor}Out[{\color{outcolor}3}]:}
402387260077093773543702433923003985719374864210714632543799910429938512
398629020592044208486969404800479988610197196058631666872994808558901323
829669944590997424504087073759918823627727188732519779505950995276120874
975462497043601418278094646496291056393887437886487337119181045825783647
849977012476632889835955735432513185323958463075557409114262417474349347
553428646576611667797396668820291207379143853719588249808126867838374559
731746136085379534524221586593201928090878297308431392844403281231558611
036976801357304216168747609675871348312025478589320767169132448426236131
412508780208000261683151027341827977704784635868170164365024153691398281
264810213092761244896359928705114964975419909342221566832572080821333186
116811553615836546984046708975602900950537616475847728421889679646244945
160765353408198901385442487984959953319101723355556602139450399736280750
137837615307127761926849034352625200015888535147331611702103968175921510
907788019393178114194545257223865541461062892187960223838971476088506276
862967146674697562911234082439208160153780889893964518263243671616762179
168909779911903754031274622289988005195444414282012187361745992642956581
746628302955570299024324153181617210465832036786906117260158783520751516
284225540265170483304226143974286933061690897968482590125458327168226458
066526769958652682272807075781391858178889652208164348344825993266043367
660176999612831860788386150279465955131156552036093988180612138558600301
435694527224206344631797460594682573103790084024432438465657245014402821
885252470935190620929023136493273497565513958720559654228749774011413346
962715422845862377387538230483865688976461927383814900140767310446640259
899490222221765904339901886018566526485061799702356193897017860040811889
729918311021171229845901641921068884387121855646124960798722908519296819
372388642614839657382291123125024186649353143970137428531926649875337218
940694281434118520158014123344828015051399694290153483077644569099073152
433278288269864602789864321139083506217095002597389863554277196742822248
757586765752344220207573630569498825087968928162753848863396909959826280
956121450994871701244516461260379029309120889086942028510640182154399457
156805941872748998094254742173582401063677404595741785160829230135358081
840096996372524230560855903700624271243416909004153690105933983835777939
410970027753472000000000000000000000000000000000000000000000000000000000
000000000000000000000000000000000000000000000000000000000000000000000000
000000000000000000000000000000000000000000000000000000000000000000000000
000000000000000000000000000000000000000000000000
\end{Verbatim}
            
The following command will stop the interpreter if executed. It is not
useful inside a \blue{Jupyter} notebook. 
Hence, the next line should not be evaluated. Therefore, I have put a
comment character ``\#'' in the first column of this line.

However, if you do remove the comment character and then evaluate the
line, nothing bad will happen as the interpreter is just restarted by
\blue{Jupyter}.

\begin{Verbatim}[commandchars=\\\{\}]
{\color{incolor}In [{\color{incolor}4}]:} \PY{c+c1}{\PYZsh{} exit()}
\end{Verbatim}

In order to write something to the screen, we can use the function
print. This function can print objects of any type. In the following
example, this function prints a string. In \textsl{Python} any character
sequence enclosed in single quotes is string.

\begin{Verbatim}[commandchars=\\\{\}]
{\color{incolor}In [{\color{incolor}5}]:} \PY{n+nb}{print}\PY{p}{(}\PY{l+s+s1}{\PYZsq{}}\PY{l+s+s1}{Hello, World!}\PY{l+s+s1}{\PYZsq{}}\PY{p}{)}
\end{Verbatim}


\begin{Verbatim}[commandchars=\\\{\}]
Hello, World!
\end{Verbatim}

Instead of using single quotes we can also use double quotes as seen in
the next example.

\begin{Verbatim}[commandchars=\\\{\}]
{\color{incolor}In [{\color{incolor}6}]:} \PY{n+nb}{print}\PY{p}{(}\PY{l+s+s2}{\PYZdq{}}\PY{l+s+s2}{Hello, World!}\PY{l+s+s2}{\PYZdq{}}\PY{p}{)}
\end{Verbatim}


\begin{Verbatim}[commandchars=\\\{\}]
Hello, World!
\end{Verbatim}

The function print accepts any number of arguments. For example, to
print the string
\\[0.2cm]
\hspace*{1.3cm}
"36 * 37 / 2 = "
\\[0.2cm]
followed by the value of the expression \(36 \cdot 37 / 2\) we can use the following print statement:

    \begin{Verbatim}[commandchars=\\\{\}]
{\color{incolor}In [{\color{incolor}7}]:} \PY{n+nb}{print}\PY{p}{(}\PY{l+s+s2}{\PYZdq{}}\PY{l+s+s2}{36 * 37 / 2 =}\PY{l+s+s2}{\PYZdq{}}\PY{p}{,} \PY{l+m+mi}{36} \PY{o}{*} \PY{l+m+mi}{37} \PY{o}{/}\PY{o}{/} \PY{l+m+mi}{2}\PY{p}{)}
\end{Verbatim}

\begin{Verbatim}[commandchars=\\\{\}]
36 * 37 / 2 = 666
\end{Verbatim}

In the expression ``36 * 37 // 2'' we have used the operator ``//'' in order
to enforce \blue{integer division}.  \index{integer division, \texttt{//}}
If we had used the operator ``/''
instead, \textsl{Python} would have used \blue{floating point division}
and therefore would have printed the floating point number 666.0 instead
of the integer 666.

\begin{Verbatim}[commandchars=\\\{\}]
{\color{incolor}In [{\color{incolor}8}]:} \PY{n+nb}{print}\PY{p}{(}\PY{l+s+s2}{\PYZdq{}}\PY{l+s+s2}{36 * 37 / 2 =}\PY{l+s+s2}{\PYZdq{}}\PY{p}{,} \PY{l+m+mi}{36} \PY{o}{*} \PY{l+m+mi}{37} \PY{o}{/} \PY{l+m+mi}{2}\PY{p}{)}
\end{Verbatim}

\begin{Verbatim}[commandchars=\\\{\}]
36 * 37 / 2 = 666.0
\end{Verbatim}

\noindent
The following script reads a natural number \(n\) and computes the sum
\(\sum\limits_{i=1}^n i\).
\begin{enumerate}
\item The function \texttt{input} prompts the user to enter a string.
\item This string is then converted into an integer using the function \texttt{int}.
\item Next, the \blue{set} \texttt{s} is created such that 
      $$\texttt{s} = \{1, \cdots, n\}. $$  
      The set \texttt{s} is constructed using the function \texttt{range}.  A function call 
      of the form $\texttt{range}(a, b + 1)$ returns a \blue{generator} 
      \index{range, $\texttt{range}(a, b+1)$}
      that produces the natural numbers 
      from $a$ to $b$.  By using this generator as an argument to the function \texttt{set}, a set is created 
      that contains all the natural number starting from $a$ upto and including $b$.
      The precise mechanics of \blue{generators} will be explained later.
\item The \texttt{print} statement uses the function \texttt{sum} to add up all the elements of the
      set \texttt{s} and print the resulting sum.
\end{enumerate}

\begin{Verbatim}[commandchars=\\\{\}]
{\color{incolor}In [{\color{incolor}9}]:} \PY{n}{n} \PY{o}{=} \PY{n+nb}{input}\PY{p}{(}\PY{l+s+s1}{\PYZsq{}}\PY{l+s+s1}{Type a natural number and press return: }\PY{l+s+s1}{\PYZsq{}}\PY{p}{)}
        \PY{n}{n} \PY{o}{=} \PY{n+nb}{int}\PY{p}{(}\PY{n}{n}\PY{p}{)}
        \PY{n}{s} \PY{o}{=} \PY{n+nb}{set}\PY{p}{(}\PY{n+nb}{range}\PY{p}{(}\PY{l+m+mi}{1}\PY{p}{,} \PY{n}{n}\PY{o}{+}\PY{l+m+mi}{1}\PY{p}{)}\PY{p}{)}
        \PY{n+nb}{print}\PY{p}{(}\PY{l+s+s1}{\PYZsq{}}\PY{l+s+s1}{The sum 1 + 2 + ... + }\PY{l+s+s1}{\PYZsq{}}\PY{p}{,} \PY{n}{n}\PY{p}{,} \PY{l+s+s1}{\PYZsq{}}\PY{l+s+s1}{ is equal to }\PY{l+s+s1}{\PYZsq{}}\PY{p}{,} \PY{n+nb}{sum}\PY{p}{(}\PY{n}{s}\PY{p}{)}\PY{p}{,} \PY{l+s+s1}{\PYZsq{}}\PY{l+s+s1}{.}\PY{l+s+s1}{\PYZsq{}}\PY{p}{,} \PY{n}{sep}\PY{o}{=} \PY{l+s+s1}{\PYZsq{}}\PY{l+s+s1}{\PYZsq{}}\PY{p}{)}
\end{Verbatim}

\begin{Verbatim}[commandchars=\\\{\}]
Type a natural number and press return: 36
The sum 1 + 2 + {\ldots} + 36 is equal to 666.
\end{Verbatim}

The following example shows how \blue{functions} can be defined in
\textsl{Python}. The function \(\texttt{sum}(n)\) is supposed to compute
the sum of all the numbers in the set \(\{1, \cdots, n\}\). Therefore, we have
 \[\texttt{sum}(n) = \sum\limits_{i=1}^n i. \]
The function sum is defined \blue{recursively}. The recursive
implementation of the function sum can best by understood if we observe
that it satisfies the following two equations:
\begin{enumerate}
\item $\texttt{sum}(0) = 0$, 
\item $\texttt{sum}(n) = \texttt{sum}(n-1) + n \quad$  provided that $n > 0$.
\end{enumerate}

    \begin{Verbatim}[commandchars=\\\{\}]
{\color{incolor}In [{\color{incolor}10}]:} \PY{k}{def} \PY{n+nf}{sum}\PY{p}{(}\PY{n}{n}\PY{p}{)}\PY{p}{:}
             \PY{k}{if} \PY{n}{n} \PY{o}{==} \PY{l+m+mi}{0}\PY{p}{:}
                 \PY{k}{return} \PY{l+m+mi}{0}
             \PY{k}{return} \PY{n+nf}{sum}\PY{p}{(}\PY{n}{n}\PY{o}{\PYZhy{}}\PY{l+m+mi}{1}\PY{p}{)} \PY{o}{+} \PY{n}{n}
\end{Verbatim}
 Let us discuss the implementation of the function sum line by line:

\begin{enumerate}
\item The keyword \texttt{\PY{k}{def}} \index{def, function definition} starts the \blue{definition} of the
      function. 
      It is followed by the \blue{name}
      of the function that is defined.  In this case, the function has the name \texttt{sum}.
      The name is followed by the list of the \blue{parameters} \index{parameter} of the function.  This list is enclosed in
      parentheses. If there is more than one parameter, the parameters have to be separated by commas.
      Finally, there needs to be a colon at the end of the first line. 
\item The \blue{body} of the function is indented.   \textbf{Contrary} to most other programming languages, 
     \textsl{Python} \blue{\underline{is s}p\underline{ace sensitive}} and indentation matters.
    
     The first statement of the body is a \blue{conditional} statement, which starts with the keyword
     \texttt{\PY{k}{if}}.  The keyword is followed by a test.  In this case we test whether the variable $n$ 
     is equal to the number $0$.  Note that this test is followed by a colon.
\item The next line contains a \texttt{\PY{k}{return}} statement.  Note that this statement is again indented.
     All statements indented by the same amount that follow an \texttt{\PY{k}{if}}-statement are considered to be
     the \blue{body} of this \texttt{\PY{k}{if}}-statement, i.e. they get executed if the test of the
     \texttt{\PY{k}{if}}-statement is true.  In this case the body contains only a single statement.

\item The last line of the function definition contains the recursive invocation of the function \texttt{sum}.
\end{enumerate}
Using the function \texttt{sum}, we can compute the sum \(\sum\limits_{i=1}^n i\) for any natural number $n$ 
as follows:
\begin{Verbatim}[commandchars=\\\{\}]
{\color{incolor}In [{\color{incolor}11}]:} \PY{n}{n}     \PY{o}{=} \PY{n+nb}{int}\PY{p}{(}\PY{n+nb}{input}\PY{p}{(}\PY{l+s+s2}{\PYZdq{}}\PY{l+s+s2}{Enter a natural number: }\PY{l+s+s2}{\PYZdq{}}\PY{p}{)}\PY{p}{)}
         \PY{n}{total} \PY{o}{=} \PY{n+nb}{sum}\PY{p}{(}\PY{n}{n}\PY{p}{)}
         \PY{k}{if} \PY{n}{n} \PY{o}{\PYZgt{}} \PY{l+m+mi}{2}\PY{p}{:}
             \PY{n+nb}{print}\PY{p}{(}\PY{l+s+s2}{\PYZdq{}}\PY{l+s+s2}{0 + 1 + 2 + ... + }\PY{l+s+s2}{\PYZdq{}}\PY{p}{,} \PY{n}{n}\PY{p}{,} \PY{l+s+s2}{\PYZdq{}}\PY{l+s+s2}{ = }\PY{l+s+s2}{\PYZdq{}}\PY{p}{,} \PY{n}{total}\PY{p}{,} \PY{n}{sep}\PY{o}{=}\PY{l+s+s1}{\PYZsq{}}\PY{l+s+s1}{\PYZsq{}}\PY{p}{)}
         \PY{k}{else}\PY{p}{:} 
             \PY{n+nb}{print}\PY{p}{(}\PY{n}{total}\PY{p}{)}
\end{Verbatim}


\begin{Verbatim}[commandchars=\\\{\}]
Enter a natural number: 100
0 + 1 + 2 + {\ldots} + 100 = 5050
\end{Verbatim}

\subsection{\texorpdfstring{Sets in
\textsl{Python}}{Sets in Python}}\label{sets-in-python}
\textsl{Python} supports \blue{sets} as a \textbf{native} datatype. This is one of the reasons that have
lead me to choose \textsl{Python} as the programming language for this
course. To get a first impression how sets are handled in \textsl{Python},
let us define two simple sets \(A\) and \(B\) and print them:

\begin{Verbatim}[commandchars=\\\{\}]
{\color{incolor}In [{\color{incolor}12}]:} \PY{n}{A} \PY{o}{=} \PY{p}{\PYZob{}}\PY{l+m+mi}{1}\PY{p}{,} \PY{l+m+mi}{2}\PY{p}{,} \PY{l+m+mi}{3}\PY{p}{\PYZcb{}}
         \PY{n}{B} \PY{o}{=} \PY{p}{\PYZob{}}\PY{l+m+mi}{2}\PY{p}{,} \PY{l+m+mi}{3}\PY{p}{,} \PY{l+m+mi}{4}\PY{p}{\PYZcb{}}
         \PY{n+nb}{print}\PY{p}{(}\PY{l+s+s1}{\PYZsq{}}\PY{l+s+s1}{A = }\PY{l+s+s1}{\PYZsq{}}\PY{p}{,} \PY{n}{A}\PY{p}{,} \PY{l+s+s1}{\PYZsq{}}\PY{l+s+s1}{, B = }\PY{l+s+s1}{\PYZsq{}}\PY{p}{,} \PY{n}{B}\PY{p}{,} \PY{n}{sep}\PY{o}{=}\PY{l+s+s1}{\PYZsq{}}\PY{l+s+s1}{\PYZsq{}}\PY{p}{)}
\end{Verbatim}


\begin{Verbatim}[commandchars=\\\{\}]
A = \{1, 2, 3\}, B = \{2, 3, 4\}
\end{Verbatim}

The last argument \texttt{sep=''} prevents the print statement from separating its arguments with space characters.
When defining the empty set, there is a caveat, as we cannot define the empty set using the
expression \texttt{\{\}}.  The reason is that this expression creates the empty \blue{dictionary} instead. (We will
discuss the data type of \blue{dictionaries} later.) To define the empty set, we therefore have
to use the following expression:

\begin{Verbatim}[commandchars=\\\{\}]
{\color{incolor}In [{\color{incolor}13}]:} \PY{n+nb}{set}\PY{p}{(}\PY{p}{)}
\end{Verbatim}


\begin{Verbatim}[commandchars=\\\{\}]
{\color{outcolor}Out[{\color{outcolor}13}]:} set()
\end{Verbatim}
Note that the empty set is also printed as set() in \textsl{Python} and not as \texttt{\{\}}.
Next, let us compute the union \(A \cup B\). This is done using the
function \texttt{union} or the operator ``\texttt{|}'':

\begin{Verbatim}[commandchars=\\\{\}]
{\color{incolor}In [{\color{incolor}14}]:} \PY{n}{A}\PY{o}{.}\PY{n}{union}\PY{p}{(}\PY{n}{B}\PY{p}{)}
\end{Verbatim}


\begin{Verbatim}[commandchars=\\\{\}]
{\color{outcolor}Out[{\color{outcolor}14}]:} \{1, 2, 3, 4\}
\end{Verbatim}
            
As the function union really acts like a \blue{method}, you might
suspect that it does change its first argument. Fortunately, this is not
the case, \(A\) is unchanged as you can see in the next line:

\begin{Verbatim}[commandchars=\\\{\}]
{\color{incolor}In [{\color{incolor}15}]:} \PY{n}{A}
\end{Verbatim}

\begin{Verbatim}[commandchars=\\\{\}]
{\color{outcolor}Out[{\color{outcolor}15}]:} \{1, 2, 3\}
\end{Verbatim}            
To compute the intersection \(A \cap B\), we use the function \texttt{intersection} or the operator ``\texttt{\&}'':

\begin{Verbatim}[commandchars=\\\{\}]
{\color{incolor}In [{\color{incolor}16}]:} \PY{n}{A}\PY{o}{.}\PY{n}{intersection}\PY{p}{(}\PY{n}{B}\PY{p}{)}
\end{Verbatim}

\begin{Verbatim}[commandchars=\\\{\}]
{\color{outcolor}Out[{\color{outcolor}16}]:} \{2, 3\}
\end{Verbatim}       
Again \(A\) is not changed.

\begin{Verbatim}[commandchars=\\\{\}]
{\color{incolor}In [{\color{incolor}17}]:} \PY{n}{A}
\end{Verbatim}

\begin{Verbatim}[commandchars=\\\{\}]
{\color{outcolor}Out[{\color{outcolor}17}]:} \{1, 2, 3\}
\end{Verbatim}       
The difference \(A \backslash B\) is computed using the operator ``\texttt{-}'':

\begin{Verbatim}[commandchars=\\\{\}]
{\color{incolor}In [{\color{incolor}18}]:} \PY{n}{A} \PY{o}{\PYZhy{}} \PY{n}{B}
\end{Verbatim}

\begin{Verbatim}[commandchars=\\\{\}]
{\color{outcolor}Out[{\color{outcolor}18}]:} \{1\}
\end{Verbatim}           
It is easy to test whether \(A \subseteq B\) holds:

\begin{Verbatim}[commandchars=\\\{\}]
{\color{incolor}In [{\color{incolor}19}]:} \PY{n}{A} \PY{o}{\PYZlt{}}\PY{o}{=} \PY{n}{B}
\end{Verbatim}

\begin{Verbatim}[commandchars=\\\{\}]
{\color{outcolor}Out[{\color{outcolor}19}]:} False
\end{Verbatim}           
Testing whether an object \(x\) is an element of a set \(M\), i.e. to
test, whether \(x \in M\) holds is straightforward:

\begin{Verbatim}[commandchars=\\\{\}]
{\color{incolor}In [{\color{incolor}20}]:} \PY{l+m+mi}{1} \PY{o+ow}{in} \PY{n}{A}
\end{Verbatim}

\begin{Verbatim}[commandchars=\\\{\}]
{\color{outcolor}Out[{\color{outcolor}20}]:} True
\end{Verbatim}          
On the other hand, the number \(1\) is not an element of the set \(B\),
i.e. we have \(1 \not\in B\):

\begin{Verbatim}[commandchars=\\\{\}]
{\color{incolor}In [{\color{incolor}21}]:} \PY{l+m+mi}{1} \PY{o+ow}{not} \PY{o+ow}{in} \PY{n}{B}
\end{Verbatim}


\begin{Verbatim}[commandchars=\\\{\}]
{\color{outcolor}Out[{\color{outcolor}21}]:} True
\end{Verbatim}
            
\subsection{Defining Sets via Selection and
Images}\label{defining-sets-via-selection-and-images}
Remember that we can define subsets of a given set \(M\) via the axiom of
selection. If \(p\) is a property such that for any object \(x\) from
the set \(M\) the expression \(p(x)\) is either True or False, the
subset of all those elements of \(M\) such that \(p(x)\) is True can be
defined as
 \[ \{ x \in M \mid p(x) \}. \] 
For example, if \(M\) is the
set \(\{1, \cdots, 100\}\) and we want to compute the subset of this set
that contains all numbers from \(M\) that are divisible by \(7\), then
this set can be defined as 
\[ \{ x \in M \mid x \;\texttt{\%}\; 7 = 0 \}. \]
In \textsl{Python}, the definition of this set can be given as follows:

    \begin{Verbatim}[commandchars=\\\{\}]
{\color{incolor}In [{\color{incolor}22}]:} \PY{n}{M} \PY{o}{=} \PY{n+nb}{set}\PY{p}{(}\PY{n+nb}{range}\PY{p}{(}\PY{l+m+mi}{1}\PY{p}{,} \PY{l+m+mi}{101}\PY{p}{)}\PY{p}{)}
         \PY{p}{\PYZob{}} \PY{n}{x} \PY{k}{for} \PY{n}{x} \PY{o+ow}{in} \PY{n}{M} \PY{k}{if} \PY{n}{x} \PY{o}{\PYZpc{}} \PY{l+m+mi}{7} \PY{o}{==} \PY{l+m+mi}{0} \PY{p}{\PYZcb{}}
\end{Verbatim}


\begin{Verbatim}[commandchars=\\\{\}]
{\color{outcolor}Out[{\color{outcolor}22}]:} \{7, 14, 21, 28, 35, 42, 49, 56, 63, 70, 77, 84, 91, 98\}
\end{Verbatim}
In general, in \textsl{Python} the set
 \[ \{ x \in M \mid p(x) \} \] 
is computed by the expression 
\[ \{\; x\; \texttt{for}\; x\; \texttt{in}\; M\; \texttt{if}\; p(x)\; \}. \]
\blue{Image} sets can be computed in a similar way. If \(f\) is a
function defined for all elements of a set \(M\), the image set
\[ \{ f(x) \mid x \in M \} \] can be computed in \textsl{Python} as
follows: \[ \{\; f(x)\; \texttt{for}\; x\; \texttt{in}\; M\; \}. \] For
example, the following expression computes the set of all squares of
numbers from the set \(\{1,\cdots,10\}\):

    \begin{Verbatim}[commandchars=\\\{\}]
{\color{incolor}In [{\color{incolor}23}]:} \PY{n}{M} \PY{o}{=} \PY{n+nb}{set}\PY{p}{(}\PY{n+nb}{range}\PY{p}{(}\PY{l+m+mi}{1}\PY{p}{,}\PY{l+m+mi}{11}\PY{p}{)}\PY{p}{)}
         \PY{p}{\PYZob{}} \PY{n}{x}\PY{o}{*}\PY{n}{x} \PY{k}{for} \PY{n}{x} \PY{o+ow}{in} \PY{n}{M} \PY{p}{\PYZcb{}}
\end{Verbatim}


\begin{Verbatim}[commandchars=\\\{\}]
{\color{outcolor}Out[{\color{outcolor}23}]:} \{1, 4, 9, 16, 25, 36, 49, 64, 81, 100\}
\end{Verbatim}
            
The computation of image sets and selections can be combined. If \(M\)
is a set, \(p\) is a property such that \(p(x)\) is either True or False
for elements of \(M\), and \(f\) is a function such that \(f(x)\) is
defined for all \(x \in M\) then we can compute set
\[ \{ f(x) \mid  x \in M \wedge p(x) \} \] of all images \(f(x)\) from
those \(x\in M\) that satisfy the property \(p(x)\) via the expression
\[ \{\; f(x)\; \texttt{for}\; x\; \texttt{in}\; M\; \texttt{if}\; p(x)\; \}. \]
For example, to compute the set of those squares of numbers from the set
\(\{1,\cdots,10\}\) that are even we can write

    \begin{Verbatim}[commandchars=\\\{\}]
{\color{incolor}In [{\color{incolor}24}]:} \PY{n}{M} \PY{o}{=} \PY{n+nb}{set}\PY{p}{(}\PY{n+nb}{range}\PY{p}{(}\PY{l+m+mi}{1}\PY{p}{,}\PY{l+m+mi}{11}\PY{p}{)}\PY{p}{)}
         \PY{p}{\PYZob{}} \PY{n}{x}\PY{o}{*}\PY{n}{x} \PY{k}{for} \PY{n}{x} \PY{o+ow}{in} \PY{n}{M} \PY{k}{if} \PY{n}{x} \PY{o}{\PYZpc{}} \PY{l+m+mi}{2} \PY{o}{==} \PY{l+m+mi}{0} \PY{p}{\PYZcb{}}
\end{Verbatim}


\begin{Verbatim}[commandchars=\\\{\}]
{\color{outcolor}Out[{\color{outcolor}24}]:} \{4, 16, 36, 64, 100\}
\end{Verbatim}
            
We can iterate over more than one set. For example, let us define the
set of all products \(p \cdot q\) of numbers \(p\) and \(q\) from the
set \(\{2, \cdots, 10\}\), i.e.~we intend to define the set
\[ \bigl\{ p \cdot q \bigm| p \in \{2,\cdots,10\} \wedge q \in \{2,\cdots,10\} \bigr\}. \]
In \textsl{Python}, this set is defined as follows:

\begin{Verbatim}[commandchars=\\\{\}]
{\color{incolor}In [{\color{incolor}25}]:} \PY{n+nb}{print}\PY{p}{(}\PY{p}{\PYZob{}} \PY{n}{p} \PY{o}{*} \PY{n}{q} \PY{k}{for} \PY{n}{p} \PY{o+ow}{in} \PY{n+nb}{range}\PY{p}{(}\PY{l+m+mi}{2}\PY{p}{,}\PY{l+m+mi}{11}\PY{p}{)} \PY{k}{for} \PY{n}{q} \PY{o+ow}{in} \PY{n+nb}{range}\PY{p}{(}\PY{l+m+mi}{2}\PY{p}{,}\PY{l+m+mi}{11}\PY{p}{)} \PY{p}{\PYZcb{}}\PY{p}{)}
\end{Verbatim}

\begin{Verbatim}[commandchars=\\\{\}]
\{4, 6, 8, 9, 10, 12, 14, 15, 16, 18, 20, 21, 24, 25, 27, 28, 30, 32, 35,
 36, 40, 42, 45, 48, 49, 50, 54, 56, 60, 63, 64, 70, 72, 80, 81, 90, 100\}
\end{Verbatim}
We can use this set to compute the set of \blue{prime numbers}. 
\index{prime numbers, \\$\{ 2, 3, 5, 7, 11, 13, 17, 19, \cdots\}$}
After all, the set of prime numbers is the set of all those natural numbers
bigger than \(1\) that can not be written as a proper product, that is a
number \(x\) is \blue{prime} if

\begin{enumerate}
\item $x$ is bigger than $1$ and 
\item there are no natural numbers $x$ and $y$ both bigger than $1$ such that $x = p * q$ holds.
\end{enumerate}
More formally, the set \(\mathbb{P}\) of prime numbers is defined as
follows:
\[ \mathbb{P} = \bigl\{ x \in \mathbb{N} \;\bigm|\; x > 1 \wedge \neg \exists p, q \in \mathbb{N}: (x = p \cdot q \wedge p > 1 \wedge q > 1)\bigr\}. \]
Hence the following code computes the set of all primes less than 100:

\begin{Verbatim}[commandchars=\\\{\}]
{\color{incolor}In [{\color{incolor}26}]:} \PY{n}{s} \PY{o}{=} \PY{n+nb}{set}\PY{p}{(}\PY{n+nb}{range}\PY{p}{(}\PY{l+m+mi}{2}\PY{p}{,}\PY{l+m+mi}{100}\PY{p}{)}\PY{p}{)}
         \PY{n+nb}{print}\PY{p}{(}\PY{n}{s} \PY{o}{\PYZhy{}} \PY{p}{\PYZob{}} \PY{n}{p} \PY{o}{*} \PY{n}{q} \PY{k}{for} \PY{n}{p} \PY{o+ow}{in} \PY{n}{s} \PY{k}{for} \PY{n}{q} \PY{o+ow}{in} \PY{n}{s} \PY{p}{\PYZcb{}}\PY{p}{)}
\end{Verbatim}


\begin{Verbatim}[commandchars=\\\{\}]
\{ 2, 3, 5, 7, 11, 13, 17, 19, 23, 29, 31, 37, 41, 43, 47, 53, 59, 61,
  67, 71, 73, 79, 83, 89, 97
\}
\end{Verbatim}
An alternative way to compute primes works by noting that a number \(p\)
is prime iff\footnote{In mathematics it is common to write ``iff'' as an abbreviation for ``if and only if''}
there is no number \(t\) other than \(1\) and \(p\) that
divides the number \(p\). The function dividers given below computes the
set of all numbers dividing a given number \(p\) evenly:

\begin{Verbatim}[commandchars=\\\{\}]
{\color{incolor}In [{\color{incolor}27}]:} \PY{k}{def} \PY{n+nf}{dividers}\PY{p}{(}\PY{n}{p}\PY{p}{)}\PY{p}{:}
             \PY{l+s+s2}{\PYZdq{}}\PY{l+s+s2}{Compute the set of numbers that divide the number p.}\PY{l+s+s2}{\PYZdq{}}
             \PY{k}{return} \PY{p}{\PYZob{}} \PY{n}{t} \PY{k}{for} \PY{n}{t} \PY{o+ow}{in} \PY{n+nb}{range}\PY{p}{(}\PY{l+m+mi}{1}\PY{p}{,} \PY{n}{p}\PY{o}{+}\PY{l+m+mi}{1}\PY{p}{)} \PY{k}{if} \PY{n}{p} \PY{o}{\PYZpc{}} \PY{n}{t} \PY{o}{==} \PY{l+m+mi}{0} \PY{p}{\PYZcb{}}
         
         \PY{n}{n}      \PY{o}{=} \PY{l+m+mi}{100}\PY{p}{;}
         \PY{n}{primes} \PY{o}{=} \PY{p}{\PYZob{}} \PY{n}{p} \PY{k}{for} \PY{n}{p} \PY{o+ow}{in} \PY{n+nb}{range}\PY{p}{(}\PY{l+m+mi}{2}\PY{p}{,} \PY{n}{n}\PY{p}{)} \PY{k}{if} \PY{n}{dividers}\PY{p}{(}\PY{n}{p}\PY{p}{)} \PY{o}{==} \PY{p}{\PYZob{}}\PY{l+m+mi}{1}\PY{p}{,} \PY{n}{p}\PY{p}{\PYZcb{}} \PY{p}{\PYZcb{}}
         \PY{n+nb}{print}\PY{p}{(}\PY{n}{primes}\PY{p}{)}
\end{Verbatim}

\begin{Verbatim}[commandchars=\\\{\}]
\{2,3,5,7,11,13,17,19,23,29,31,37,41,43,47,53,59,61,67,71,73,79,83,89,97\}
\end{Verbatim}

\subsection{Computing the Power Set}\label{computing-the-power-set}
\index{power set, \\ computing the power set in \textsl{Python}}
Unfortunately, there is no operator to compute the power set \(2^M\) of
a given set \(M\). Since the power set is needed frequently, we have to
implement a function \texttt{power} to compute this set ourselves. The easiest
way to compute the power set \(2^M\) of a set \(M\) is to implement the
following recursive equations:
\begin{enumerate}
\item The power set of the empty set contains only the empty set:
    $$2^{\{\}} = \bigl\{\{\}\bigr\}$$
\item If a set $M$ can be written as $M = C \cup \{x\}$, where the element $x$ does not occur in the set $C$, then the power set $2^M$ consists of two sets:
      \begin{itemize}
        \item Firstly, all subsets of $C$ are also subsets of $M$.
        \item Secondly, if A is a subset of $C$, then the set $A \cup\{x\}$ is also a subset of $M$.
      \end{itemize}
    If we combine these parts we get the following equation:
    $$2^{C \cup \{x\}} = 2^C \cup \bigl\{ A \cup \{x\} \bigm| A \in 2^C \bigr\}$$
\end{enumerate}
But there is another problem: In \textsl{Python} we can't create a set
that has elements that are sets themselves! The reason is that in
\textsl{Python} sets are implemented via \blue{hash tables} and therefore
the elements of a set need to be \blue{hashable}.  \index{hashable}
(The notion of an
element being \blue{hashable} will be discussed in more detail in the
lecture on \blue{Algorithms}.) However, sets are \blue{mutable} \index{mutable}, i.e.~sets can be changed,
and \blue{mutable} objects are not \blue{hashable}. Fortunately, there is a
workaround: \textsl{Python} provides the data type of \blue{frozen sets}.
\index{frozen set}
These sets behave like sets but are lacking certain functions that modify sets and
hence are \blue{immutable}. So if we use \blue{frozen sets} as elements of the
power set, we can compute the power set of a given set. The function
\texttt{power} given below shows how this works.

\begin{Verbatim}[commandchars=\\\{\}]
{\color{incolor}In [{\color{incolor}28}]:} \PY{k}{def} \PY{n+nf}{power}\PY{p}{(}\PY{n}{M}\PY{p}{)}\PY{p}{:}
             \PY{l+s+s2}{\PYZdq{}}\PY{l+s+s2}{This function computes the power set of the set M.}\PY{l+s+s2}{\PYZdq{}}
             \PY{k}{if} \PY{n}{M} \PY{o}{==} \PY{n+nb}{set}\PY{p}{(}\PY{p}{)}\PY{p}{:}
                 \PY{k}{return} \PY{p}{\PYZob{}} \PY{n+nb}{frozenset}\PY{p}{(}\PY{p}{)} \PY{p}{\PYZcb{}}
             \PY{k}{else}\PY{p}{:}
                 \PY{n}{C}  \PY{o}{=} \PY{n+nb}{set}\PY{p}{(}\PY{n}{M}\PY{p}{)}  \PY{c+c1}{\PYZsh{} C is a copy of M as we don\PYZsq{}t want to change the set M}
                 \PY{n}{x}  \PY{o}{=} \PY{n}{C}\PY{o}{.}\PY{n}{pop}\PY{p}{(}\PY{p}{)} \PY{c+c1}{\PYZsh{} pop removes an element x from the set C}
                 \PY{n}{P1} \PY{o}{=} \PY{n}{power}\PY{p}{(}\PY{n}{C}\PY{p}{)}
                 \PY{n}{P2} \PY{o}{=} \PY{p}{\PYZob{}} \PY{n}{A}\PY{o}{.}\PY{n}{union}\PY{p}{(}\PY{p}{\PYZob{}}\PY{n}{x}\PY{p}{\PYZcb{}}\PY{p}{)} \PY{k}{for} \PY{n}{A} \PY{o+ow}{in} \PY{n}{P1} \PY{p}{\PYZcb{}}
                 \PY{k}{return} \PY{n}{P1}\PY{o}{.}\PY{n}{union}\PY{p}{(}\PY{n}{P2}\PY{p}{)}
\end{Verbatim}


\begin{Verbatim}[commandchars=\\\{\}]
{\color{incolor}In [{\color{incolor}29}]:} \PY{n}{power}\PY{p}{(}\PY{n}{A}\PY{p}{)}
\end{Verbatim}


\begin{Verbatim}[commandchars=\\\{\}]
{\color{outcolor}Out[{\color{outcolor}29}]:} \{frozenset(),
          frozenset(\{3\}),
          frozenset(\{1\}),
          frozenset(\{2\}),
          frozenset(\{1, 2\}),
          frozenset(\{2, 3\}),
          frozenset(\{1, 3\}),
          frozenset(\{1, 2, 3\})\}
\end{Verbatim}
Let us print this in a more readable way. To this end we implement a
function \texttt{prettify} that turns a set of frozensets into a string that
looks like a set of sets.

\begin{Verbatim}[commandchars=\\\{\}]
{\color{incolor}In [{\color{incolor}30}]:} \PY{k}{def} \PY{n+nf}{prettify}\PY{p}{(}\PY{n}{M}\PY{p}{)}\PY{p}{:}
             \PY{l+s+sd}{\PYZdq{}\PYZdq{}\PYZdq{}Turn the set of frozen sets M into a string that looks like a set of sets.}
         \PY{l+s+sd}{       M is assumed to be the power set of some set.}
         \PY{l+s+sd}{    \PYZdq{}\PYZdq{}\PYZdq{}}
             \PY{n}{result} \PY{o}{=} \PY{l+s+s2}{\PYZdq{}}\PY{l+s+s2}{\PYZob{}\PYZob{}}\PY{l+s+s2}{\PYZcb{}, }\PY{l+s+s2}{\PYZdq{}}   \PY{c+c1}{\PYZsh{} The emepty set is always an element of a power set.}
             \PY{k}{for} \PY{n}{A} \PY{o+ow}{in} \PY{n}{M}\PY{p}{:}
                 \PY{k}{if} \PY{n}{A} \PY{o}{==} \PY{n+nb}{set}\PY{p}{(}\PY{p}{)}\PY{p}{:} \PY{c+c1}{\PYZsh{} The empty set has already been taken care of.}
                     \PY{k}{continue}
                 \PY{n}{result} \PY{o}{+}\PY{o}{=} \PY{n+nb}{str}\PY{p}{(}\PY{n+nb}{set}\PY{p}{(}\PY{n}{A}\PY{p}{)}\PY{p}{)} \PY{o}{+} \PY{l+s+s2}{\PYZdq{}}\PY{l+s+s2}{, }\PY{l+s+s2}{\PYZdq{}} \PY{c+c1}{\PYZsh{} A is converted from a frozen set to a set}
             \PY{n}{result} \PY{o}{=} \PY{n}{result}\PY{p}{[}\PY{p}{:}\PY{o}{\PYZhy{}}\PY{l+m+mi}{2}\PY{p}{]} \PY{c+c1}{\PYZsh{} remove the trailing substring \PYZdq{}, \PYZdq{}}
             \PY{n}{result} \PY{o}{+}\PY{o}{=} \PY{l+s+s2}{\PYZdq{}}\PY{l+s+s2}{\PYZcb{}}\PY{l+s+s2}{\PYZdq{}}
             \PY{k}{return} \PY{n}{result}
\end{Verbatim}


\begin{Verbatim}[commandchars=\\\{\}]
{\color{incolor}In [{\color{incolor}31}]:} \PY{n}{prettify}\PY{p}{(}\PY{n}{power}\PY{p}{(}\PY{n}{A}\PY{p}{)}\PY{p}{)}
\end{Verbatim}


\begin{Verbatim}[commandchars=\\\{\}]
{\color{outcolor}Out[{\color{outcolor}31}]:} '\{\{\}, \{3\}, \{1, 2\}, \{2, 3\}, \{1\}, \{1, 3\}, \{1, 2, 3\}, \{2\}\}'
\end{Verbatim}
            
\subsection{Pairs and Cartesian
Products}\label{pairs-and-cartesian-products}
\index{pairs}
In \textsl{Python}, pairs can be created by enclosing the components of
the pair in parentheses. For example, to compute the pair
\(\langle 1, 2 \rangle\) we can write:

\begin{Verbatim}[commandchars=\\\{\}]
{\color{incolor}In [{\color{incolor}32}]:} \PY{p}{(}\PY{l+m+mi}{1}\PY{p}{,}\PY{l+m+mi}{2}\PY{p}{)}
\end{Verbatim}


\begin{Verbatim}[commandchars=\\\{\}]
{\color{outcolor}Out[{\color{outcolor}32}]:} (1, 2)
\end{Verbatim}
It is not even necessary to enclose the components of a pair in
parentheses. For example, to compute the pair \(\langle 1, 2 \rangle\)
we can use the following expression:

\begin{Verbatim}[commandchars=\\\{\}]
{\color{incolor}In [{\color{incolor}33}]:} \PY{l+m+mi}{1}\PY{p}{,} \PY{l+m+mi}{2}
\end{Verbatim}

\begin{Verbatim}[commandchars=\\\{\}]
{\color{outcolor}Out[{\color{outcolor}33}]:} (1, 2)
\end{Verbatim}
The Cartesian product \(A \times B\) of two sets \(A\) and \(B\) can now
be computed via the following expression:
\[ \{\; (x, y) \;\texttt{for}\; x \;\texttt{in}\; A\; \texttt{for}\; y\; \texttt{in}\; B\; \} \]
For example, as we have defined \(A\) as \(\{1,2,3\}\) and \(B\) as
\(\{2,3,4\}\), the Cartesian product of \(A\) and \(B\) is computed as
follows:

\begin{Verbatim}[commandchars=\\\{\}]
{\color{incolor}In [{\color{incolor}34}]:} \PY{p}{\PYZob{}} \PY{p}{(}\PY{n}{x}\PY{p}{,}\PY{n}{y}\PY{p}{)} \PY{k}{for} \PY{n}{x} \PY{o+ow}{in} \PY{n}{A} \PY{k}{for} \PY{n}{y} \PY{o+ow}{in} \PY{n}{B} \PY{p}{\PYZcb{}}
\end{Verbatim}


\begin{Verbatim}[commandchars=\\\{\}]
{\color{outcolor}Out[{\color{outcolor}34}]:} \{(1, 2), (1, 3), (1, 4), (2, 2), (2, 3), (2, 4), (3, 2), (3, 3), (3, 4)\}
\end{Verbatim}
            
\subsection{Tuples}\label{tuples}
\index{tuple}
The notion of a tuple is a generalization of the notion of a pair. For example, to compute the tuple
\(\langle 1, 2, 3 \rangle\) we can use the following expression:

\begin{Verbatim}[commandchars=\\\{\}]
{\color{incolor}In [{\color{incolor}35}]:} \PY{p}{(}\PY{l+m+mi}{1}\PY{p}{,} \PY{l+m+mi}{2}\PY{p}{,} \PY{l+m+mi}{3}\PY{p}{)}
\end{Verbatim}

\begin{Verbatim}[commandchars=\\\{\}]
{\color{outcolor}Out[{\color{outcolor}35}]:} (1, 2, 3)
\end{Verbatim}
Longer tuples can be build using the function range in combination with the function tuple:

\begin{Verbatim}[commandchars=\\\{\}]
{\color{incolor}In [{\color{incolor}36}]:} \PY{n+nb}{tuple}\PY{p}{(}\PY{n+nb}{range}\PY{p}{(}\PY{l+m+mi}{1}\PY{p}{,} \PY{l+m+mi}{11}\PY{p}{)}\PY{p}{)}
\end{Verbatim}


\begin{Verbatim}[commandchars=\\\{\}]
{\color{outcolor}Out[{\color{outcolor}36}]:} (1, 2, 3, 4, 5, 6, 7, 8, 9, 10)
\end{Verbatim}
Tuples can be \blue{concatenated} using the operator ``\texttt{+}'':

\begin{Verbatim}[commandchars=\\\{\}]
{\color{incolor}In [{\color{incolor}37}]:} \PY{n}{T1} \PY{o}{=} \PY{p}{(}\PY{l+m+mi}{1}\PY{p}{,} \PY{l+m+mi}{2}\PY{p}{,} \PY{l+m+mi}{3}\PY{p}{)}
         \PY{n}{T2} \PY{o}{=} \PY{p}{(}\PY{l+m+mi}{4}\PY{p}{,} \PY{l+m+mi}{5}\PY{p}{,} \PY{l+m+mi}{6}\PY{p}{)}
         \PY{n}{T3} \PY{o}{=} \PY{n}{T1} \PY{o}{+} \PY{n}{T2}
         \PY{n}{T3}
\end{Verbatim}


\begin{Verbatim}[commandchars=\\\{\}]
{\color{outcolor}Out[{\color{outcolor}37}]:} (1, 2, 3, 4, 5, 6)
\end{Verbatim}
The \blue{length} of a tuple is computed using the function \texttt{len}:

\begin{Verbatim}[commandchars=\\\{\}]
{\color{incolor}In [{\color{incolor}38}]:} \PY{n+nb}{len}\PY{p}{(}\PY{n}{T3}\PY{p}{)}
\end{Verbatim}


\begin{Verbatim}[commandchars=\\\{\}]
{\color{outcolor}Out[{\color{outcolor}38}]:} 6
\end{Verbatim}
            
The components of a tuple can be extracted using square brackets. Note
that the first component actually has the index \(0\)! This is similar
to the behaviour of \blue{arrays} in the programming language C.

\begin{Verbatim}[commandchars=\\\{\}]
{\color{incolor}In [{\color{incolor}39}]:} \PY{n+nb}{print}\PY{p}{(}\PY{l+s+s2}{\PYZdq{}}\PY{l+s+s2}{T3[0] =}\PY{l+s+s2}{\PYZdq{}}\PY{p}{,} \PY{n}{T3}\PY{p}{[}\PY{l+m+mi}{0}\PY{p}{]}\PY{p}{)}
         \PY{n+nb}{print}\PY{p}{(}\PY{l+s+s2}{\PYZdq{}}\PY{l+s+s2}{T3[1] =}\PY{l+s+s2}{\PYZdq{}}\PY{p}{,} \PY{n}{T3}\PY{p}{[}\PY{l+m+mi}{1}\PY{p}{]}\PY{p}{)}
         \PY{n+nb}{print}\PY{p}{(}\PY{l+s+s2}{\PYZdq{}}\PY{l+s+s2}{T3[2] =}\PY{l+s+s2}{\PYZdq{}}\PY{p}{,} \PY{n}{T3}\PY{p}{[}\PY{l+m+mi}{2}\PY{p}{]}\PY{p}{)}
\end{Verbatim}

\begin{Verbatim}[commandchars=\\\{\}]
T3[0] = 1
T3[1] = 2
T3[2] = 3
\end{Verbatim}
If we use negative indices, then we index from the back@www of the tuple, as
shown in the following example:

\begin{Verbatim}[commandchars=\\\{\}]
{\color{incolor}In [{\color{incolor}40}]:} \PY{n+nb}{print}\PY{p}{(}\PY{l+s+s2}{\PYZdq{}}\PY{l+s+s2}{T3[\PYZhy{}1] =}\PY{l+s+s2}{\PYZdq{}}\PY{p}{,} \PY{n}{T3}\PY{p}{[}\PY{o}{\PYZhy{}}\PY{l+m+mi}{1}\PY{p}{]}\PY{p}{)} \PY{c+c1}{\PYZsh{} last element}
         \PY{n+nb}{print}\PY{p}{(}\PY{l+s+s2}{\PYZdq{}}\PY{l+s+s2}{T3[\PYZhy{}2] =}\PY{l+s+s2}{\PYZdq{}}\PY{p}{,} \PY{n}{T3}\PY{p}{[}\PY{o}{\PYZhy{}}\PY{l+m+mi}{2}\PY{p}{]}\PY{p}{)} \PY{c+c1}{\PYZsh{} penultimate element}
         \PY{n+nb}{print}\PY{p}{(}\PY{l+s+s2}{\PYZdq{}}\PY{l+s+s2}{T3[\PYZhy{}3] =}\PY{l+s+s2}{\PYZdq{}}\PY{p}{,} \PY{n}{T3}\PY{p}{[}\PY{o}{\PYZhy{}}\PY{l+m+mi}{3}\PY{p}{]}\PY{p}{)} \PY{c+c1}{\PYZsh{} third last element }
\end{Verbatim}

\begin{Verbatim}[commandchars=\\\{\}]
T3[-1] = 6
T3[-2] = 5
T3[-3] = 4
\end{Verbatim}

\begin{Verbatim}[commandchars=\\\{\}]
{\color{incolor}In [{\color{incolor}41}]:} \PY{n}{T3}
\end{Verbatim}

\begin{Verbatim}[commandchars=\\\{\}]
{\color{outcolor}Out[{\color{outcolor}41}]:} (1, 2, 3, 4, 5, 6)
\end{Verbatim}
            
The \blue{slicing} \index{slicing, $L[a:b]$}
operator extracts a subtuple from a given tuple. If
\(L\) is a tuple and \(a\) and \(b\) are natural numbers such that
\(a \leq b\) and \(a,b \in \{0, \texttt{len}(L) \}\), then the syntax of
the slicing operator is as follows: \[ L[a:b] \] The expression
\(L[a:b]\) extracts the subtuple that starts with the element \(L[a]\)
up to and excluding the element \(L[b]\). The following shows an
example:

\begin{Verbatim}[commandchars=\\\{\}]
{\color{incolor}In [{\color{incolor}42}]:} \PY{n}{L} \PY{o}{=} \PY{n+nb}{tuple}\PY{p}{(}\PY{n+nb}{range}\PY{p}{(}\PY{l+m+mi}{1}\PY{p}{,}\PY{l+m+mi}{11}\PY{p}{)}\PY{p}{)}
         \PY{n}{L}\PY{p}{[}\PY{l+m+mi}{2}\PY{p}{:}\PY{l+m+mi}{6}\PY{p}{]}
\end{Verbatim}


\begin{Verbatim}[commandchars=\\\{\}]
{\color{outcolor}Out[{\color{outcolor}42}]:} (3, 4, 5, 6)
\end{Verbatim}
            
Slicing works with negative indices, too:

\begin{Verbatim}[commandchars=\\\{\}]
{\color{incolor}In [{\color{incolor}43}]:} \PY{n}{L}\PY{p}{[}\PY{l+m+mi}{2}\PY{p}{:}\PY{o}{\PYZhy{}}\PY{l+m+mi}{2}\PY{p}{]}
\end{Verbatim}

\begin{Verbatim}[commandchars=\\\{\}]
{\color{outcolor}Out[{\color{outcolor}43}]:} (3, 4, 5, 6, 7, 8)
\end{Verbatim}
If we want to create a tuple of length  $1$, we have to use the following syntax:
\\[0.2cm]
\hspace*{1.3cm}
\texttt{L = (x,)}
\\[0.2cm]
Note that in the expression above the comma is not optional as the expression $(x)$ would be interpreted as 
$x$. 

\subsection{Lists}\label{lists}
\index{list}
Next, we discuss the data type of lists. Lists are a lot like tuples,
but in contrast to tuples, lists are \blue{mutatable}, i.e.~we can
change lists. To construct a list, we use square backets:

\begin{Verbatim}[commandchars=\\\{\}]
{\color{incolor}In [{\color{incolor}44}]:} \PY{n}{L} \PY{o}{=} \PY{p}{[}\PY{l+m+mi}{1}\PY{p}{,}\PY{l+m+mi}{2}\PY{p}{,}\PY{l+m+mi}{3}\PY{p}{]}
         \PY{n}{L}
\end{Verbatim}


\begin{Verbatim}[commandchars=\\\{\}]
{\color{outcolor}Out[{\color{outcolor}44}]:} [1, 2, 3]
\end{Verbatim}
To change the first element of a list, we can use the \blue{index operator}:

\begin{Verbatim}[commandchars=\\\{\}]
{\color{incolor}In [{\color{incolor}45}]:} \PY{n}{L}\PY{p}{[}\PY{l+m+mi}{0}\PY{p}{]} \PY{o}{=} \PY{l+m+mi}{7}
         \PY{n}{L}
\end{Verbatim}

\begin{Verbatim}[commandchars=\\\{\}]
{\color{outcolor}Out[{\color{outcolor}45}]:} [7, 2, 3]
\end{Verbatim}
This last operation would not be possible if L had been a tuple instead
of a list. Lists support concatenation in the same way as tuples:

\begin{Verbatim}[commandchars=\\\{\}]
{\color{incolor}In [{\color{incolor}46}]:} \PY{p}{[}\PY{l+m+mi}{1}\PY{p}{,}\PY{l+m+mi}{2}\PY{p}{,}\PY{l+m+mi}{3}\PY{p}{]} \PY{o}{+} \PY{p}{[}\PY{l+m+mi}{4}\PY{p}{,}\PY{l+m+mi}{5}\PY{p}{,}\PY{l+m+mi}{6}\PY{p}{]}
\end{Verbatim}


\begin{Verbatim}[commandchars=\\\{\}]
{\color{outcolor}Out[{\color{outcolor}46}]:} [1, 2, 3, 4, 5, 6]
\end{Verbatim}
The function \texttt{len} computes the length of a list:

\begin{Verbatim}[commandchars=\\\{\}]
{\color{incolor}In [{\color{incolor}47}]:} \PY{n+nb}{len}\PY{p}{(}\PY{p}{[}\PY{l+m+mi}{4}\PY{p}{,}\PY{l+m+mi}{5}\PY{p}{,}\PY{l+m+mi}{6}\PY{p}{]}\PY{p}{)}
\end{Verbatim}

\begin{Verbatim}[commandchars=\\\{\}]
{\color{outcolor}Out[{\color{outcolor}47}]:} 3
\end{Verbatim}
            
Lists and tuples both support the functions \texttt{max} and \texttt{min}. The expression
\(\texttt{max}(L)\) computes the maximum of all the elements of the list
(or tuple) \(L\), while \(\texttt{min}(L)\) computes the smallest
element of \(L\).

\begin{Verbatim}[commandchars=\\\{\}]
{\color{incolor}In [{\color{incolor}48}]:} \PY{n+nb}{max}\PY{p}{(}\PY{p}{[}\PY{l+m+mi}{1}\PY{p}{,}\PY{l+m+mi}{2}\PY{p}{,}\PY{l+m+mi}{3}\PY{p}{]}\PY{p}{)}
\end{Verbatim}


\begin{Verbatim}[commandchars=\\\{\}]
{\color{outcolor}Out[{\color{outcolor}48}]:} 3
\end{Verbatim}
            
\begin{Verbatim}[commandchars=\\\{\}]
{\color{incolor}In [{\color{incolor}49}]:} \PY{n+nb}{min}\PY{p}{(}\PY{p}{[}\PY{l+m+mi}{1}\PY{p}{,}\PY{l+m+mi}{2}\PY{p}{,}\PY{l+m+mi}{3}\PY{p}{]}\PY{p}{)}
\end{Verbatim}


\begin{Verbatim}[commandchars=\\\{\}]
{\color{outcolor}Out[{\color{outcolor}49}]:} 1
\end{Verbatim}
The functions ``\texttt{min}'' and ``\texttt{max}'' also work for sets.

\subsection{Boolean Operators}\label{boolean-operators}

In \textsl{Python}, the \emph{Boolean} values are written as \texttt{True} and \texttt{False}.

\begin{Verbatim}[commandchars=\\\{\}]
{\color{incolor}In [{\color{incolor}50}]:} \PY{k+kc}{True}
\end{Verbatim}

\begin{Verbatim}[commandchars=\\\{\}]
{\color{outcolor}Out[{\color{outcolor}50}]:} True
\end{Verbatim}
            
\begin{Verbatim}[commandchars=\\\{\}]
{\color{incolor}In [{\color{incolor}51}]:} \PY{k+kc}{False}
\end{Verbatim}

\begin{Verbatim}[commandchars=\\\{\}]
{\color{outcolor}Out[{\color{outcolor}51}]:} False
\end{Verbatim}
            
    These values can be combined using the Boolean operator ``\(\wedge\)'',
``\(\vee\)'', and ``\(\neg\)''. In \textsl{Python}, these operators are denoted as
``\texttt{and}'', ``\texttt{or}'', and ``\texttt{not}''. The following table shows how the operator ``\texttt{and}`` is
defined:

\begin{Verbatim}[commandchars=\\\{\}]
{\color{incolor}In [{\color{incolor}52}]:} \PY{n}{B} \PY{o}{=} \PY{p}{(}\PY{k+kc}{True}\PY{p}{,} \PY{k+kc}{False}\PY{p}{)}
         \PY{k}{for} \PY{n}{x} \PY{o+ow}{in} \PY{n}{B}\PY{p}{:}
             \PY{k}{for} \PY{n}{y} \PY{o+ow}{in} \PY{n}{B}\PY{p}{:}
                 \PY{n+nb}{print}\PY{p}{(}\PY{n}{x}\PY{p}{,} \PY{l+s+s1}{\PYZsq{}}\PY{l+s+s1}{and}\PY{l+s+s1}{\PYZsq{}}\PY{p}{,} \PY{n}{y}\PY{p}{,} \PY{l+s+s1}{\PYZsq{}}\PY{l+s+s1}{=}\PY{l+s+s1}{\PYZsq{}}\PY{p}{,} \PY{n}{x} \PY{o+ow}{and} \PY{n}{y}\PY{p}{)}
\end{Verbatim}

\begin{Verbatim}[commandchars=\\\{\}]
True and True = True
True and False = False
False and True = False
False and False = False
\end{Verbatim}
Next, we show the table for the operator ``\texttt{or}''.
The disjunction of two Boolean values is only \texttt{False} if both values are
\texttt{False}:

\begin{Verbatim}[commandchars=\\\{\}]
{\color{incolor}In [{\color{incolor}53}]:} \PY{k}{for} \PY{n}{x} \PY{o+ow}{in} \PY{n}{B}\PY{p}{:}
             \PY{k}{for} \PY{n}{y} \PY{o+ow}{in} \PY{n}{B}\PY{p}{:}
                 \PY{n+nb}{print}\PY{p}{(}\PY{n}{x}\PY{p}{,} \PY{l+s+s1}{\PYZsq{}}\PY{l+s+s1}{or}\PY{l+s+s1}{\PYZsq{}}\PY{p}{,} \PY{n}{y}\PY{p}{,} \PY{l+s+s1}{\PYZsq{}}\PY{l+s+s1}{=}\PY{l+s+s1}{\PYZsq{}}\PY{p}{,} \PY{n}{x} \PY{o+ow}{or} \PY{n}{y}\PY{p}{)}
\end{Verbatim}


\begin{Verbatim}[commandchars=\\\{\}]
True or True = True
True or False = True
False or True = True
False or False = False
\end{Verbatim}
Finally, the negation operator ``\texttt{not}'' works as expected:

\begin{Verbatim}[commandchars=\\\{\}]
{\color{incolor}In [{\color{incolor}54}]:} \PY{k}{for} \PY{n}{x} \PY{o+ow}{in} \PY{n}{B}\PY{p}{:}
             \PY{n+nb}{print}\PY{p}{(}\PY{l+s+s1}{\PYZsq{}}\PY{l+s+s1}{not}\PY{l+s+s1}{\PYZsq{}}\PY{p}{,} \PY{n}{x}\PY{p}{,} \PY{l+s+s1}{\PYZsq{}}\PY{l+s+s1}{=}\PY{l+s+s1}{\PYZsq{}}\PY{p}{,} \PY{o+ow}{not} \PY{n}{x}\PY{p}{)}
\end{Verbatim}

\begin{Verbatim}[commandchars=\\\{\}]
not True = False
not False = True
\end{Verbatim}
Boolean values are created by comparing numbers using the following comparison operators:
\begin{enumerate}
\item $a\;\texttt{==}\;b$ is true iff $a$ is equal to $b$.
\item $a\;\texttt{!=}\;b$ is true iff $a$ is different from $b$.
\item $a\;\texttt{<}\;b$ is true iff $a$ is less than $b$.
\item $a\;\texttt{<=}\;b$ is true iff $a$ is less than or equal to $b$.
\item $a\;\texttt{>=}\;b$ is true iff $a$ is bigger than or equal to $b$.
\item $a\;\texttt{>}\;b$ is true iff $a$ is bigger than $b$.
\end{enumerate}

\begin{Verbatim}[commandchars=\\\{\}]
{\color{incolor}In [{\color{incolor}55}]:} \PY{l+m+mi}{1} \PY{o}{==} \PY{l+m+mi}{2}
\end{Verbatim}


\begin{Verbatim}[commandchars=\\\{\}]
{\color{outcolor}Out[{\color{outcolor}55}]:} False
\end{Verbatim}
            
\begin{Verbatim}[commandchars=\\\{\}]
{\color{incolor}In [{\color{incolor}56}]:} \PY{l+m+mi}{1} \PY{o}{\PYZlt{}} \PY{l+m+mi}{2}
\end{Verbatim}


\begin{Verbatim}[commandchars=\\\{\}]
{\color{outcolor}Out[{\color{outcolor}56}]:} True
\end{Verbatim}
            
\begin{Verbatim}[commandchars=\\\{\}]
{\color{incolor}In [{\color{incolor}57}]:} \PY{l+m+mi}{1} \PY{o}{\PYZlt{}}\PY{o}{=} \PY{l+m+mi}{2}
\end{Verbatim}


\begin{Verbatim}[commandchars=\\\{\}]
{\color{outcolor}Out[{\color{outcolor}57}]:} True
\end{Verbatim}
            
\begin{Verbatim}[commandchars=\\\{\}]
{\color{incolor}In [{\color{incolor}58}]:} \PY{l+m+mi}{1} \PY{o}{\PYZgt{}} \PY{l+m+mi}{2}
\end{Verbatim}


\begin{Verbatim}[commandchars=\\\{\}]
{\color{outcolor}Out[{\color{outcolor}58}]:} False
\end{Verbatim}
            
\begin{Verbatim}[commandchars=\\\{\}]
{\color{incolor}In [{\color{incolor}59}]:} \PY{l+m+mi}{1} \PY{o}{\PYZgt{}}\PY{o}{=} \PY{l+m+mi}{2}
\end{Verbatim}


\begin{Verbatim}[commandchars=\\\{\}]
{\color{outcolor}Out[{\color{outcolor}59}]:} False
\end{Verbatim}
Comparison operators can be \blue{chained} as shown in the following example:

\begin{Verbatim}[commandchars=\\\{\}]
{\color{incolor}In [{\color{incolor}60}]:} \PY{l+m+mi}{1} \PY{o}{\PYZlt{}} \PY{l+m+mi}{2} \PY{o}{\PYZlt{}} \PY{l+m+mi}{3}
\end{Verbatim}


\begin{Verbatim}[commandchars=\\\{\}]
{\color{outcolor}Out[{\color{outcolor}60}]:} True
\end{Verbatim}
            
\textsl{Python} supports the \blue{universal quantifier} \(\forall\) (read: \emph{for all}). If \(L\) is
a list of Boolean values, then we can check whether all elements of
\(L\) are true by writing \[ \texttt{all}(L) \] For example, to check
whether all elements of a list \(L\) are even we can write the
following:

\begin{Verbatim}[commandchars=\\\{\}]
{\color{incolor}In [{\color{incolor}61}]:} \PY{n}{L} \PY{o}{=} \PY{p}{[}\PY{l+m+mi}{2}\PY{p}{,} \PY{l+m+mi}{4}\PY{p}{,} \PY{l+m+mi}{6}\PY{p}{]}
         \PY{n+nb}{all}\PY{p}{(}\PY{p}{[}\PY{n}{x} \PY{o}{\PYZpc{}} \PY{l+m+mi}{2} \PY{o}{==} \PY{l+m+mi}{0} \PY{k}{for} \PY{n}{x} \PY{o+ow}{in} \PY{n}{L}\PY{p}{]}\PY{p}{)}
\end{Verbatim}


\begin{Verbatim}[commandchars=\\\{\}]
{\color{outcolor}Out[{\color{outcolor}61}]:} True
\end{Verbatim}
            
    \subsection{Control Structures}\label{control-structures}

    First of all, \textsl{Python} supports \emph{branching} statements. The following
example is taken from the \textsl{Python} tutorial at \href{https://python.org}{https://python.org}:

\begin{Verbatim}[commandchars=\\\{\}]
{\color{incolor}In [{\color{incolor}62}]:} \PY{n}{x} \PY{o}{=} \PY{n+nb}{int}\PY{p}{(}\PY{n+nb}{input}\PY{p}{(}\PY{l+s+s2}{\PYZdq{}}\PY{l+s+s2}{Please enter an integer: }\PY{l+s+s2}{\PYZdq{}}\PY{p}{)}\PY{p}{)}
         \PY{k}{if} \PY{n}{x} \PY{o}{\PYZlt{}} \PY{l+m+mi}{0}\PY{p}{:}
            \PY{n+nb}{print}\PY{p}{(}\PY{l+s+s1}{\PYZsq{}}\PY{l+s+s1}{The number is negative!}\PY{l+s+s1}{\PYZsq{}}\PY{p}{)}
         \PY{k}{elif} \PY{n}{x} \PY{o}{==} \PY{l+m+mi}{0}\PY{p}{:}
            \PY{n+nb}{print}\PY{p}{(}\PY{l+s+s1}{\PYZsq{}}\PY{l+s+s1}{The number is zero.}\PY{l+s+s1}{\PYZsq{}}\PY{p}{)}
         \PY{k}{elif} \PY{n}{x} \PY{o}{==} \PY{l+m+mi}{1}\PY{p}{:}
            \PY{n+nb}{print}\PY{p}{(}\PY{l+s+s2}{\PYZdq{}}\PY{l+s+s2}{It}\PY{l+s+s2}{\PYZsq{}}\PY{l+s+s2}{s a one.}\PY{l+s+s2}{\PYZdq{}}\PY{p}{)}
         \PY{k}{else}\PY{p}{:}
            \PY{n+nb}{print}\PY{p}{(}\PY{l+s+s2}{\PYZdq{}}\PY{l+s+s2}{It}\PY{l+s+s2}{\PYZsq{}}\PY{l+s+s2}{s more than one.}\PY{l+s+s2}{\PYZdq{}}\PY{p}{)}
\end{Verbatim}

\begin{Verbatim}[commandchars=\\\{\}]
Please enter an integer: 42
It's more than one.
\end{Verbatim}
\blue{Loops} can be used to iterate over sets, lists, tuples, or
generators. The following example prints the numbers from 1 to 10.
\index{loop} \index{for}

\begin{Verbatim}[commandchars=\\\{\}]
{\color{incolor}In [{\color{incolor}63}]:} \PY{k}{for} \PY{n}{x} \PY{o+ow}{in} \PY{n+nb}{range}\PY{p}{(}\PY{l+m+mi}{1}\PY{p}{,} \PY{l+m+mi}{11}\PY{p}{)}\PY{p}{:}
             \PY{n+nb}{print}\PY{p}{(}\PY{n}{x}\PY{p}{)}
\end{Verbatim}


\begin{Verbatim}[commandchars=\\\{\}]
1
2
3
4
5
6
7
8
9
10
\end{Verbatim}
The same can be achieved with a \texttt{while} loop:
\index{while}

\begin{Verbatim}[commandchars=\\\{\}]
{\color{incolor}In [{\color{incolor}64}]:} \PY{n}{x} \PY{o}{=} \PY{l+m+mi}{1}
         \PY{k}{while} \PY{n}{x} \PY{o}{\PYZlt{}}\PY{o}{=} \PY{l+m+mi}{7}\PY{p}{:}
             \PY{n+nb}{print}\PY{p}{(}\PY{n}{x}\PY{p}{)}
             \PY{n}{x} \PY{o}{+}\PY{o}{=} \PY{l+m+mi}{1}
\end{Verbatim}


\begin{Verbatim}[commandchars=\\\{\}]
1
2
3
4
5
6
7
\end{Verbatim}
The following program computes the prime numbers according to an
algorithm given by  \href{https://en.wikipedia.org/wiki/Eratosthenes}{(Eratosthenes of Cyrene, 276 BC -- 195/194 BC)}.
\index{Eratosthenes of Cyrene}

\begin{enumerate}
\item We set $n$ equal to 100 as we want to compute the set all prime numbers less or equal that 100.
\item \texttt{primes} is the list of numbers from 0 upto $n$, i.e.~we have initially
      $$ \texttt{primes} = [0,1,2,\cdots,n] $$
      Therefore, we have
      $$ \texttt{primes}[i] = i \quad \mbox{for all $i \in \{0,1,\cdots,n\}$.} $$
      The idea is to set \texttt{primes[$i$]} to zero iff $i$ is a proper product of two numbers.

\item To this end we iterate over all $i$ and $j$ from the set $\{2,\cdots,n\}$
  and set the product $\texttt{primes}[i*j]$ to zero.  This is achieved by the two \texttt{for} loops below.
\item Note that we have to check that the product $i * j$ is not bigger than $n$ for otherwise we would get an
  \blue{out of range error}  when trying to assign \texttt{primes[i*j]}.
\item After the iteration, all non-prime elements greater than one of the list primes have been set to zero.
\item Finally, we compute the set of primes by collecting those elements that have not been set to $0$.
\end{enumerate}

\begin{Verbatim}[commandchars=\\\{\}]
{\color{incolor}In [{\color{incolor}65}]:} \PY{n}{n}      \PY{o}{=} \PY{l+m+mi}{100}
         \PY{n}{primes} \PY{o}{=} \PY{n+nb}{list}\PY{p}{(}\PY{n+nb}{range}\PY{p}{(}\PY{l+m+mi}{0}\PY{p}{,} \PY{n}{n}\PY{o}{+}\PY{l+m+mi}{1}\PY{p}{)}\PY{p}{)}
         \PY{k}{for} \PY{n}{i} \PY{o+ow}{in} \PY{n+nb}{range}\PY{p}{(}\PY{l+m+mi}{2}\PY{p}{,} \PY{n}{n}\PY{o}{+}\PY{l+m+mi}{1}\PY{p}{)}\PY{p}{:}
             \PY{k}{for} \PY{n}{j} \PY{o+ow}{in} \PY{n+nb}{range}\PY{p}{(}\PY{l+m+mi}{2}\PY{p}{,} \PY{n}{n}\PY{o}{+}\PY{l+m+mi}{1}\PY{p}{)}\PY{p}{:}
                 \PY{k}{if} \PY{n}{i} \PY{o}{*} \PY{n}{j} \PY{o}{\PYZlt{}}\PY{o}{=} \PY{n}{n}\PY{p}{:}
                     \PY{n}{primes}\PY{p}{[}\PY{n}{i} \PY{o}{*} \PY{n}{j}\PY{p}{]} \PY{o}{=} \PY{l+m+mi}{0}
         \PY{n+nb}{print}\PY{p}{(}\PY{n}{primes}\PY{p}{)}
         \PY{n+nb}{print}\PY{p}{(}\PY{p}{\PYZob{}} \PY{n}{i} \PY{k}{for} \PY{n}{i} \PY{o+ow}{in} \PY{n+nb}{range}\PY{p}{(}\PY{l+m+mi}{2}\PY{p}{,} \PY{n}{n}\PY{o}{+}\PY{l+m+mi}{1}\PY{p}{)} \PY{k}{if} \PY{n}{primes}\PY{p}{[}\PY{n}{i}\PY{p}{]} \PY{o}{!=} \PY{l+m+mi}{0} \PY{p}{\PYZcb{}}\PY{p}{)}
\end{Verbatim}

\begin{Verbatim}[commandchars=\\\{\}]
[0, 1, 2, 3, 0, 5, 0, 7, 0, 0, 0, 11, 0, 13, 0, 0, 0, 17, 0, 19, 0, 0, 0, 23,
 0, 0, 0, 0, 0, 29, 0, 31, 0, 0, 0, 0, 0, 37, 0, 0, 0, 41, 0, 43, 0, 0, 0, 47,
 0, 0, 0, 0, 0, 53, 0, 0, 0, 0, 0, 59, 0, 61, 0, 0, 0, 0, 0, 67, 0, 0, 0, 71,
 0, 73, 0, 0, 0, 0, 0, 79, 0, 0, 0, 83, 0, 0, 0, 0, 0, 89, 0, 0, 0, 0, 0, 0,
 0, 97, 0, 0, 0]
\{ 2, 3, 5, 7, 11, 13, 17, 19, 23, 29, 31, 37, 41, 43, 47, 53, 59, 61, 67, 71,
  73, 79, 83, 89, 97
\}
\end{Verbatim}
The algorithm given above can be improved by using the following observations:
\begin{enumerate}[(a)]
\item If a number $x$ can be written as a product $a * b$, then at least one of the numbers $a$ or $b$ has to
      be less than or equal to $\sqrt{x}$.  Therefore, the \texttt{for} loop below iterates as long as $i \leq
      \sqrt{x}$.
      The function \texttt{ceil} is needed to cast the square root of $x$ to a natural number.  In
      order to use the functions \texttt{sqrt} and \texttt{ceil} we have to import them from the module
      \texttt{math}.  \index{math, \texttt{import math}} This is done in line 1 of the program shown below.  
\item When we iterate over $j$ in the inner loop, it is sufficient if we start with $j = i$ since all products
      of the form $i * j$ where $j < i$ have already been eliminated at the time when the multiples of $i$ had
      been eliminated. 
\item If \texttt{primes[$i$] = 0}, then $i$ is not a prime and hence it has to be a product of two numbers $a$
      and $b$ both of which are smaller than $i$.  However, since all the multiples of $a$ and $b$ have already
      been eliminated, there is no point in eliminating the multiples of $i$ since these are also multiples of both
      $a$ and $b$ and hence have already been eliminated.  Therefore, if \texttt{primes[$i$] = 0} we can
      immediately jump to the next value of $i$.  This is achieved by the \texttt{continue} statement 
      below. 
\end{enumerate}
The program shown below is easily capable of computing all prime numbers less than a million.

\begin{Verbatim}[commandchars=\\\{\}]
{\color{incolor}In [{\color{incolor}66}]:} \PY{k+kn}{from} \PY{n+nn}{math} \PY{k}{import} \PY{n}{sqrt}\PY{p}{,} \PY{n}{ceil}
         
         \PY{n}{n} \PY{o}{=} \PY{l+m+mi}{1000}
         \PY{n}{primes} \PY{o}{=} \PY{n+nb}{list}\PY{p}{(}\PY{n+nb}{range}\PY{p}{(}\PY{n}{n}\PY{o}{+}\PY{l+m+mi}{1}\PY{p}{)}\PY{p}{)}
         \PY{k}{for} \PY{n}{i} \PY{o+ow}{in} \PY{n+nb}{range}\PY{p}{(}\PY{l+m+mi}{2}\PY{p}{,} \PY{n}{ceil}\PY{p}{(}\PY{n}{sqrt}\PY{p}{(}\PY{n}{n}\PY{p}{)}\PY{p}{)}\PY{p}{)}\PY{p}{:}
             \PY{k}{if} \PY{n}{primes}\PY{p}{[}\PY{n}{i}\PY{p}{]} \PY{o}{==} \PY{l+m+mi}{0}\PY{p}{:}
                 \PY{k}{continue}
             \PY{n}{j} \PY{o}{=} \PY{n}{i}
             \PY{k}{while} \PY{n}{i} \PY{o}{*} \PY{n}{j} \PY{o}{\PYZlt{}}\PY{o}{=} \PY{n}{n}\PY{p}{:}
                 \PY{n}{primes}\PY{p}{[}\PY{n}{i} \PY{o}{*} \PY{n}{j}\PY{p}{]} \PY{o}{=} \PY{l+m+mi}{0}
                 \PY{n}{j} \PY{o}{+}\PY{o}{=} \PY{l+m+mi}{1}\PY{p}{;}
         \PY{n+nb}{print}\PY{p}{(}\PY{p}{\PYZob{}} \PY{n}{i} \PY{k}{for} \PY{n}{i} \PY{o+ow}{in} \PY{n+nb}{range}\PY{p}{(}\PY{l+m+mi}{2}\PY{p}{,} \PY{n}{n}\PY{o}{+}\PY{l+m+mi}{1}\PY{p}{)} \PY{k}{if} \PY{n}{primes}\PY{p}{[}\PY{n}{i}\PY{p}{]} \PY{o}{!=} \PY{l+m+mi}{0} \PY{p}{\PYZcb{}}\PY{p}{)}
\end{Verbatim}


\begin{Verbatim}[commandchars=\\\{\}]
\{ 2, 3, 5, 7, 521, 11, 523, 13, 17, 19, 23, 29, 541, 31, 547, 37, 41, 43, 557,
  47, 563, 53, 569, 59, 571, 61, 577, 67, 71, 73, 587, 79, 593, 83, 599, 89,
  601, 607, 97, 101, 613, 103, 617, 107, 619, 109, 113, 631, 127, 641, 131,
  643, 647, 137, 139, 653, 659, 149, 661, 151, 157, 673, 163, 677, 167, 683,
  173, 179, 691, 181, 701, 191, 193, 197, 709, 199, 719, 211, 727, 733, 223,
  227, 739, 229, 743, 233, 239, 751, 241, 757, 761, 251, 257, 769, 773, 263,
  269, 271, 787, 277, 281, 283, 797, 293, 809, 811, 307, 821, 311, 823, 313,
  827, 317, 829, 839, 331, 337, 853, 857, 347, 859, 349, 863, 353, 359, 877,
  367, 881, 883, 373, 887, 379, 383, 389, 907, 397, 911, 401, 919, 409, 929,
  419, 421, 937, 941, 431, 433, 947, 439, 953, 443, 449, 967, 457, 971, 461,
  463, 977, 467, 983, 479, 991, 997, 487, 491, 499, 503, 509
\}
\end{Verbatim}

\subsection{Numerical Functions}\label{numerical-functions}

\textsl{Python} provides all of the mathematical functions that you have
come to learn at school. A detailed listing of these functions can be
found at
\href{https://docs.python.org/3.7/library/math.html}{https://docs.python.org/3.7/library/math.html}. We just
show the most important functions and constants. In order to make the module \texttt{math}
available, we use the following \blue{\texttt{import} statement}: \index{math, \texttt{import math}}

\begin{Verbatim}[commandchars=\\\{\}]
{\color{incolor}In [{\color{incolor}67}]:} \PY{k+kn}{import} \PY{n+nn}{math}
\end{Verbatim}
The mathematical constant \href{https://en.wikipedia.org/wiki/Pi}{Pi}, which is most often written as \(\pi\), is
available as \texttt{math.pi}.

\begin{Verbatim}[commandchars=\\\{\}]
{\color{incolor}In [{\color{incolor}68}]:} \PY{n}{math}\PY{o}{.}\PY{n}{pi}
\end{Verbatim}

\begin{Verbatim}[commandchars=\\\{\}]
{\color{outcolor}Out[{\color{outcolor}68}]:} 3.141592653589793
\end{Verbatim}
The \blue{sine} function is called as follows:
\begin{Verbatim}[commandchars=\\\{\}]
{\color{incolor}In [{\color{incolor}69}]:} \PY{n}{math}\PY{o}{.}\PY{n}{sin}\PY{p}{(}\PY{n}{math}\PY{o}{.}\PY{n}{pi}\PY{o}{/}\PY{l+m+mi}{6}\PY{p}{)}
\end{Verbatim}

\begin{Verbatim}[commandchars=\\\{\}]
{\color{outcolor}Out[{\color{outcolor}69}]:} 0.49999999999999994
\end{Verbatim}           
The \blue{cosine} function is called as follows: 
\begin{Verbatim}[commandchars=\\\{\}]
{\color{incolor}In [{\color{incolor}70}]:} \PY{n}{math}\PY{o}{.}\PY{n}{cos}\PY{p}{(}\PY{l+m+mf}{0.0}\PY{p}{)}
\end{Verbatim}


\begin{Verbatim}[commandchars=\\\{\}]
{\color{outcolor}Out[{\color{outcolor}70}]:} 1.0
\end{Verbatim}
The \blue{tangent} function is called as follows:
\begin{Verbatim}[commandchars=\\\{\}]
{\color{incolor}In [{\color{incolor}71}]:} \PY{n}{math}\PY{o}{.}\PY{n}{tan}\PY{p}{(}\PY{n}{math}\PY{o}{.}\PY{n}{pi}\PY{o}{/}\PY{l+m+mi}{4}\PY{p}{)}
\end{Verbatim}

\begin{Verbatim}[commandchars=\\\{\}]
{\color{outcolor}Out[{\color{outcolor}71}]:} 0.9999999999999999
\end{Verbatim}
The \blue{arc sine}, \blue{arc cosine}, and \blue{arc tangent} are called by prefixing the
character '\texttt{a}' to the name of the function as seen below:
\begin{Verbatim}[commandchars=\\\{\}]
{\color{incolor}In [{\color{incolor}72}]:} \PY{n}{math}\PY{o}{.}\PY{n}{asin}\PY{p}{(}\PY{l+m+mf}{1.0}\PY{p}{)}
\end{Verbatim}

\begin{Verbatim}[commandchars=\\\{\}]
{\color{outcolor}Out[{\color{outcolor}72}]:} 1.5707963267948966
\end{Verbatim}
            
\begin{Verbatim}[commandchars=\\\{\}]
{\color{incolor}In [{\color{incolor}73}]:} \PY{n}{math}\PY{o}{.}\PY{n}{acos}\PY{p}{(}\PY{l+m+mf}{1.0}\PY{p}{)}
\end{Verbatim}

\begin{Verbatim}[commandchars=\\\{\}]
{\color{outcolor}Out[{\color{outcolor}73}]:} 0.0
\end{Verbatim}
            
\begin{Verbatim}[commandchars=\\\{\}]
{\color{incolor}In [{\color{incolor}74}]:} \PY{n}{math}\PY{o}{.}\PY{n}{atan}\PY{p}{(}\PY{l+m+mf}{1.0}\PY{p}{)}
\end{Verbatim}

\begin{Verbatim}[commandchars=\\\{\}]
{\color{outcolor}Out[{\color{outcolor}74}]:} 0.7853981633974483
\end{Verbatim} 
\href{https://en.wikipedia.org/wiki/E_(mathematical_constant)}{Euler's number} \(e\) \index{Euler's number}
can be computed as follows:
\begin{Verbatim}[commandchars=\\\{\}]
{\color{incolor}In [{\color{incolor}75}]:} \PY{n}{math}\PY{o}{.}\PY{n}{e}
\end{Verbatim}

\begin{Verbatim}[commandchars=\\\{\}]
{\color{outcolor}Out[{\color{outcolor}75}]:} 2.718281828459045
\end{Verbatim}
The \blue{exponential} function \(\mathrm{exp}(x) := e^x\) is computed as follows:
\begin{Verbatim}[commandchars=\\\{\}]
{\color{incolor}In [{\color{incolor}76}]:} \PY{n}{math}\PY{o}{.}\PY{n}{exp}\PY{p}{(}\PY{l+m+mi}{1}\PY{p}{)}
\end{Verbatim}

\begin{Verbatim}[commandchars=\\\{\}]
{\color{outcolor}Out[{\color{outcolor}76}]:} 2.718281828459045
\end{Verbatim}
The \blue{natural logarithm} \(\ln(x)\), which is defined as the inverse function of the function \(\exp(x)\), is called \texttt{log} (instead of \texttt{ln}):
\begin{Verbatim}[commandchars=\\\{\}]
{\color{incolor}In [{\color{incolor}77}]:} \PY{n}{math}\PY{o}{.}\PY{n}{log}\PY{p}{(}\PY{n}{math}\PY{o}{.}\PY{n}{e} \PY{o}{*} \PY{n}{math}\PY{o}{.}\PY{n}{e}\PY{p}{)}
\end{Verbatim}

\begin{Verbatim}[commandchars=\\\{\}]
{\color{outcolor}Out[{\color{outcolor}77}]:} 2.0
\end{Verbatim}
The \blue{square root} \(\sqrt{x}\) of a number \(x\) is computed using the
function \texttt{sqrt}:
\begin{Verbatim}[commandchars=\\\{\}]
{\color{incolor}In [{\color{incolor}78}]:} \PY{n}{math}\PY{o}{.}\PY{n}{sqrt}\PY{p}{(}\PY{l+m+mi}{2}\PY{p}{)}
\end{Verbatim}

\begin{Verbatim}[commandchars=\\\{\}]
{\color{outcolor}Out[{\color{outcolor}78}]:} 1.4142135623730951
\end{Verbatim}


\subsection{Selection Sort}
In order to see a practical application of the concepts discussed so far, we present a \blue{sorting algorithm}
that is known as \href{https://en.wikipedia.org/wiki/Selection_sort}{\blue{selection sort}}
(Deutsch: \blue{Sortieren durch Auswahl}).  \index{selection sort}
This algorithm sorts a given list \texttt{L} and works as follows:
\begin{enumerate}
\item If \texttt{L} is empty, \texttt{sort(L)} is also empty:
      \\[0.2cm]
      \hspace*{1.3cm}
      $\texttt{sort([])} = \texttt{[]}$.
\item Otherwise, we first compute the minimum of \texttt{L}.  Clearly, the minimum needs to be the
      first element of the sorted list.  We remove this minimum from \texttt{L}, sort the remaining
      elements recursively, and finally attach the minimum at the front of this list:
      \\[0.2cm]
      \hspace*{1.3cm}
      $\texttt{sort(L)} = \texttt{[min(L)] + sort([}x \in \texttt{L} \texttt{|} x \not= \texttt{min}(L)\texttt{])}$.
\end{enumerate}
Figure \ref{fig:min-sort.py} on page \pageref{fig:min-sort.py} shows the program
\href{https://github.com/karlstroetmann/Logic/blob/master/Python/min-sort.py}{\texttt{min-sort.py}}
that implements selection sort  in \textsl{Python}. 

\begin{figure}[!ht]
\centering
\begin{Verbatim}[ frame         = lines, 
                  framesep      = 0.3cm, 
                  labelposition = bottomline,
                  numbers       = left,
                  numbersep     = -0.2cm,
                  xleftmargin   = 0.8cm,
                  xrightmargin  = 0.8cm,
                ]
    def minSort(L):
        if L == []:
            return []
        m = min(L)
        return [m] + minSort([x for x in L if x != m])
    
    L = [ 2, 13, 5, 13, 7, 2, 4 ]
    print('minSort(', L, ') = ', minSort(L), sep='')
\end{Verbatim}
\vspace*{-0.3cm}
\caption{Implementing selection sort in \textsl{Python}.}
\label{fig:min-sort.py}
\end{figure}


\section{Loading a Program}
The \textsl{Python} interpreter can \blue{load} programs interactively into a running session.
If \textsl{file} is the base name of a file, then the command
\\[0.2cm]
\hspace*{1.3cm}
\texttt{import }\textsl{file}
\\[0.2cm]
loads the program from  \textsl{file}\texttt{.py} and executes the statements given in this program.
For example, the command
\\[0.2cm]
\hspace*{1.3cm}
\texttt{import min\_sort}
\\[0.2cm]
executes the program shown in Figure
\ref{fig:min-sort.py} on page \pageref{fig:min-sort.py}.  If we want to call a function defined in the file
\texttt{min\_sort.py},  then we have to prefix this function as shown below:
\\[0.2cm]
\hspace*{1.3cm}
\texttt{min\_sort.minSort([2, 13, 5, 13, 7, 2, 4])},
\\[0.2cm]
i.e.~we have to prefix the name of the function that we want to call with the base name of the file defining
this function followed by a dot character.

\section{Strings}
\index{Strings}
\textsl{Python} support \blue{strings}.  \href{https://en.wikipedia.org/wiki/String_(computer_science)}{Strings} are
nothing more but sequences of characters.  In \textsl{Python}, these have to be 
enclosed either in double quotes or in single quotes.  The operator ``\texttt{+}'' can be used to concatenate
strings.  For example, the expression 
\\[0.2cm]
\hspace*{1.3cm}
\texttt{\squote{abc} + \texttt{'uvw'};}
\\[0.2cm]
returns the result
\\[0.2cm]
\hspace*{1.3cm}
\squote{abcuvw}.
\\[0.2cm]
Furthermore, a natural number \texttt{n} can be multiplied with a string \texttt{s}.  The expression
\\[0.2cm]
\hspace*{1.3cm}
\texttt{n * s;}
\\[0.2cm]
returns a string consisting of \texttt{n} concatenations of \texttt{s}.  For example,
the result of
\\[0.2cm]
\hspace*{1.3cm}
\texttt{3 * \squote{abc};}
\\[0.2cm]
is the string \squote{abcabcabc}.  When multiplying a string with a number, the order of the
arguments does not matter. Hence, the expression
\\[0.2cm]
\hspace*{1.3cm}
\texttt{\squote{abc} * 3}
\\[0.2cm]
also yields the result \squote{abcabcabc}.  In order to extract substrings from a given string, we can use the same
slicing operator that also works for lists and tuples.  Therefore, if $s$ is a string and $k$ and $l$ are numbers, then
the expression 
\\[0.2cm]
\hspace*{1.3cm}
\texttt{$s$[$k$..$l$]}
\\[0.2cm]
extracts the substring form $s$ that starts with the $k+1$th character of $s$ and that ends with the $l$th character.
For example, if \texttt{s} is defined by the assignment
\\[0.2cm]
\hspace*{1.3cm}
\texttt{s = \symbol{34}abcdefgh\symbol{34}}
\\[0.2cm]
then the expression \texttt{s[2:5]} returns the substring
\\[0.2cm]
\hspace*{1.3cm}
\texttt{\symbol{34}cde\symbol{34}}.

\section{Computing with Unlimited Precision}
\index{fractions}
\textsl{Python} provides the module fractions that implements
\blue{rational numbers} through the function Fraction that is
implemented in this module. We can load this function as follows:
\begin{Verbatim}[commandchars=\\\{\}]
{\color{incolor}In [{\color{incolor}1}]:} \PY{k+kn}{from} \PY{n+nn}{fractions} \PY{k}{import} \PY{n}{Fraction}
\end{Verbatim}
The function Fraction expects two arguments, the \blue{nominator} and
the \blue{denominator}. Mathematically, we have
\[ \texttt{Fraction}(p, q) = \frac{p}{q}. \] For example, we can compute
the sum \(\frac{1}{2} + \frac{1}{3}\) as follows:
\begin{Verbatim}[commandchars=\\\{\}]
{\color{incolor}In [{\color{incolor}2}]:} \PY{n+nb}{sum} \PY{o}{=} \PY{n}{Fraction}\PY{p}{(}\PY{l+m+mi}{1}\PY{p}{,} \PY{l+m+mi}{2}\PY{p}{)} \PY{o}{+} \PY{n}{Fraction}\PY{p}{(}\PY{l+m+mi}{1}\PY{p}{,} \PY{l+m+mi}{3}\PY{p}{)}
        \PY{n+nb}{print}\PY{p}{(}\PY{n+nb}{sum}\PY{p}{)}
\end{Verbatim}


\begin{Verbatim}[commandchars=\\\{\}]
5/6
\end{Verbatim}
Let us compute Euler's number \(e\). \index{Euler's number} The easiest way to compute \(e\) is
as inifinite series. We have that
\[ e = \sum\limits_{n=0}^\infty \frac{1}{n!}. \]
Here \(n!\) denotes the
\blue{factorial} (Deutsch: \blue{Fakultät}) \index{factorial} of \(n\), which is defined as follows:
\[ n! = 1 \cdot 2 \cdot 3 \cdot {\dots} \cdot n. \]
\begin{Verbatim}[commandchars=\\\{\}]
{\color{incolor}In [{\color{incolor}3}]:} \PY{k}{def} \PY{n+nf}{factorial}\PY{p}{(}\PY{n}{n}\PY{p}{)}\PY{p}{:}
            \PY{l+s+s2}{\PYZdq{}}\PY{l+s+s2}{compute the factorial of n}\PY{l+s+s2}{\PYZdq{}}
            \PY{n}{result} \PY{o}{=} \PY{l+m+mi}{1}
            \PY{k}{for} \PY{n}{i} \PY{o+ow}{in} \PY{n+nb}{range}\PY{p}{(}\PY{l+m+mi}{1}\PY{p}{,} \PY{n}{n}\PY{o}{+}\PY{l+m+mi}{1}\PY{p}{)}\PY{p}{:}
                \PY{n}{result} \PY{o}{*}\PY{o}{=} \PY{n}{i}
            \PY{k}{return} \PY{n}{result}
\end{Verbatim}
Let's check that our definition of the factorial works as expected.
\begin{Verbatim}[commandchars=\\\{\}]
{\color{incolor}In [{\color{incolor}4}]:} \PY{k}{for} \PY{n}{i} \PY{o+ow}{in} \PY{n+nb}{range}\PY{p}{(}\PY{l+m+mi}{10}\PY{p}{)}\PY{p}{:}
            \PY{n+nb}{print}\PY{p}{(}\PY{n}{i}\PY{p}{,} \PY{l+s+s1}{\PYZsq{}}\PY{l+s+s1}{! = }\PY{l+s+s1}{\PYZsq{}}\PY{p}{,} \PY{n}{factorial}\PY{p}{(}\PY{n}{i}\PY{p}{)}\PY{p}{,} \PY{n}{sep}\PY{o}{=}\PY{l+s+s1}{\PYZsq{}}\PY{l+s+s1}{\PYZsq{}}\PY{p}{)}
\end{Verbatim}

\begin{Verbatim}[commandchars=\\\{\}]
0! = 1
1! = 1
2! = 2
3! = 6
4! = 24
5! = 120
6! = 720
7! = 5040
8! = 40320
9! = 362880
\end{Verbatim}
Lets approximate \(e\) by the following sum:
\[ e = \sum\limits_{i=0}^n \frac{1}{i!} \] Setting \(n=100\) should be
sufficient to compute \(e\) to a hundred decimal places.

\begin{Verbatim}[commandchars=\\\{\}]
{\color{incolor}In [{\color{incolor}5}]:} \PY{n}{n} \PY{o}{=} \PY{l+m+mi}{100}
\end{Verbatim}

\begin{Verbatim}[commandchars=\\\{\}]
{\color{incolor}In [{\color{incolor}6}]:} \PY{n}{e} \PY{o}{=} \PY{l+m+mi}{0}
        \PY{k}{for} \PY{n}{i} \PY{o+ow}{in} \PY{n+nb}{range}\PY{p}{(}\PY{n}{n}\PY{o}{+}\PY{l+m+mi}{1}\PY{p}{)}\PY{p}{:}
            \PY{n}{e} \PY{o}{+}\PY{o}{=} \PY{n}{Fraction}\PY{p}{(}\PY{l+m+mi}{1}\PY{p}{,} \PY{n}{factorial}\PY{p}{(}\PY{n}{i}\PY{p}{)}\PY{p}{)}
\end{Verbatim}
Multiply \(e\) by \(10^{100}\) and round so that we get the first 100
decimal places of \(e\):

\begin{Verbatim}[commandchars=\\\{\}]
{\color{incolor}In [{\color{incolor}7}]:} \PY{n}{eTimesBig} \PY{o}{=} \PY{n}{e} \PY{o}{*} \PY{l+m+mi}{10} \PY{o}{*}\PY{o}{*} \PY{n}{n}
        \PY{n}{s} \PY{o}{=} \PY{n+nb}{str}\PY{p}{(}\PY{n+nb}{round}\PY{p}{(}\PY{n}{eTimesBig}\PY{p}{)}\PY{p}{)}
\end{Verbatim}
Insert a ``\texttt{.}'' after the first digit:
\begin{Verbatim}[commandchars=\\\{\}]
{\color{incolor}In [{\color{incolor}8}]:} \PY{n+nb}{print}\PY{p}{(}\PY{n}{s}\PY{p}{[}\PY{l+m+mi}{0}\PY{p}{]}\PY{p}{,} \PY{l+s+s1}{\PYZsq{}}\PY{l+s+s1}{.}\PY{l+s+s1}{\PYZsq{}}\PY{p}{,} \PY{n}{s}\PY{p}{[}\PY{l+m+mi}{1}\PY{p}{:}\PY{p}{]}\PY{p}{,} \PY{n}{sep}\PY{o}{=}\PY{l+s+s1}{\PYZsq{}}\PY{l+s+s1}{\PYZsq{}}\PY{p}{)}
\end{Verbatim}
And there we go. Ladies and gentlemen, lo and behold: Here are the first 100 digits of $e$:
\begin{Verbatim}[commandchars=\\\{\}]
2.718281828459045235360287471352662497757247093699959574966967627724076630353547
5945713821785251664274
\end{Verbatim}

\section{Dictionaries}\label{dictionaries}
\index{dictionary, \texttt{dict}}
    A \emph{binary relation} \(R\) is a subset of the cartesian product of
two sets \(A\) and \(B\), i.e.~if \(R\) is a binary relation we have:
\[ R \subseteq A \times B \] A binary relation
\(R \subseteq A \times B\) is a \emph{functional} relation \index{functional relation}
if and only if we have: \[ \forall x \in A: \forall y_1, y_2 \in B: \bigl(
     \langle x, y_1\rangle \in R \wedge \langle x, y_2\rangle \in R
     \rightarrow y_1 = y_2
   \bigr)
 \]
 If \(R\) is a fuctional relation, then \(R \subseteq A \times B\) can
be interpreted as a function \[ f_R:A \rightarrow B \] that is defined
as follows:
\[ f_R(x) := y \quad\mbox{iff}\quad \langle x, y\rangle \in R. \]
In \emph{Python} a functional relation \(R \subseteq A \times B\) can be
represented as a \textbf{dictionary}, provided \(R\) is finite. The
empty dictionary is defined as follows:
\begin{Verbatim}[commandchars=\\\{\}]
{\color{incolor}In [{\color{incolor}2}]:} \PY{n}{emptyDict} \PY{o}{=} \PY{p}{\PYZob{}}\PY{p}{\PYZcb{}}
\end{Verbatim}
The syntax to define a functional relation \(R\) of the form
\[ \bigl\{ \langle x_1, y_1\rangle, \cdots, \langle x_n, y_n\rangle \bigr\} \]
in \textsl{Python} is as follows: We have to write
\\[0.2cm]
\hspace*{1.3cm}
\texttt{\{ x1:y1, ... xn:yn \}}
\\[0.2cm]
An example will clarify this. The dictionary Number2English maps the first nine
numbers to their English names.

\begin{Verbatim}[commandchars=\\\{\}]
{\color{incolor}In [{\color{incolor}3}]:} \PY{n}{Number2English} \PY{o}{=} \PY{p}{\PYZob{}} \PY{l+m+mi}{1}\PY{p}{:}\PY{l+s+s1}{\PYZsq{}}\PY{l+s+s1}{one}\PY{l+s+s1}{\PYZsq{}}\PY{p}{,} \PY{l+m+mi}{2}\PY{p}{:}\PY{l+s+s1}{\PYZsq{}}\PY{l+s+s1}{two}\PY{l+s+s1}{\PYZsq{}}\PY{p}{,} \PY{l+m+mi}{3}\PY{p}{:}\PY{l+s+s1}{\PYZsq{}}\PY{l+s+s1}{three}\PY{l+s+s1}{\PYZsq{}}\PY{p}{,} \PY{l+m+mi}{4}\PY{p}{:}\PY{l+s+s1}{\PYZsq{}}\PY{l+s+s1}{four}\PY{l+s+s1}{\PYZsq{}}\PY{p}{,} \PY{l+m+mi}{5}\PY{p}{:}\PY{l+s+s1}{\PYZsq{}}\PY{l+s+s1}{five}\PY{l+s+s1}{\PYZsq{}}\PY{p}{,} 
                           \PY{l+m+mi}{6}\PY{p}{:}\PY{l+s+s1}{\PYZsq{}}\PY{l+s+s1}{six}\PY{l+s+s1}{\PYZsq{}}\PY{p}{,} \PY{l+m+mi}{7}\PY{p}{:}\PY{l+s+s1}{\PYZsq{}}\PY{l+s+s1}{seven}\PY{l+s+s1}{\PYZsq{}}\PY{p}{,} \PY{l+m+mi}{8}\PY{p}{:}\PY{l+s+s1}{\PYZsq{}}\PY{l+s+s1}{eight}\PY{l+s+s1}{\PYZsq{}}\PY{p}{,} \PY{l+m+mi}{9}\PY{p}{:}\PY{l+s+s1}{\PYZsq{}}\PY{l+s+s1}{nine}\PY{l+s+s1}{\PYZsq{}}
                         \PY{p}{\PYZcb{}}
        \PY{n}{Number2English}
\end{Verbatim}


\begin{Verbatim}[commandchars=\\\{\}]
{\color{outcolor}Out[{\color{outcolor}3}]:} \{1: 'one',
         2: 'two',
         3: 'three',
         4: 'four',
         5: 'five',
         6: 'six',
         7: 'seven',
         8: 'eight',
         9: 'nine'\}
\end{Verbatim}
Here, the numbers \(1, \cdots, 9\) are called the \emph{keys} of the dictionary.

We can use the dictionary Number2English as if it were a function: If we
write \(\texttt{Number2English[}k\texttt{]}\), then this expression will
return the name of the number \(k\) provided \(k \in \{1,\cdots,9\}\).

\begin{Verbatim}[commandchars=\\\{\}]
{\color{incolor}In [{\color{incolor}4}]:} \PY{n}{Number2English}\PY{p}{[}\PY{l+m+mi}{2}\PY{p}{]}
\end{Verbatim}

\begin{Verbatim}[commandchars=\\\{\}]
{\color{outcolor}Out[{\color{outcolor}4}]:} 'two'
\end{Verbatim}       
The expression Number2English{[}10{]} would return an error message, as
\(10\) is not a key of the dictionary Number2English. We can check
whether an object is a key of a dictionary by using the operator in as
shown below:
\begin{Verbatim}[commandchars=\\\{\}]
{\color{incolor}In [{\color{incolor}5}]:} \PY{l+m+mi}{10} \PY{o+ow}{in} \PY{n}{Number2English}
\end{Verbatim}

\begin{Verbatim}[commandchars=\\\{\}]
{\color{outcolor}Out[{\color{outcolor}5}]:} False
\end{Verbatim}
            
\begin{Verbatim}[commandchars=\\\{\}]
{\color{incolor}In [{\color{incolor}6}]:} \PY{l+m+mi}{7} \PY{o+ow}{in} \PY{n}{Number2English}
\end{Verbatim}

\begin{Verbatim}[commandchars=\\\{\}]
{\color{outcolor}Out[{\color{outcolor}6}]:} True
\end{Verbatim}
We can easily extend our dictionary as shown below:
\begin{Verbatim}[commandchars=\\\{\}]
{\color{incolor}In [{\color{incolor}7}]:} \PY{n}{Number2English}\PY{p}{[}\PY{l+m+mi}{10}\PY{p}{]} \PY{o}{=} \PY{l+s+s1}{\PYZsq{}}\PY{l+s+s1}{ten}\PY{l+s+s1}{\PYZsq{}}
\end{Verbatim}

\begin{Verbatim}[commandchars=\\\{\}]
{\color{incolor}In [{\color{incolor}8}]:} \PY{n}{Number2English}
\end{Verbatim}

\begin{Verbatim}[commandchars=\\\{\}]
{\color{outcolor}Out[{\color{outcolor}8}]:} \{1: 'one',
         2: 'two',
         3: 'three',
         4: 'four',
         5: 'five',
         6: 'six',
         7: 'seven',
         8: 'eight',
         9: 'nine',
         10: 'ten'\}
\end{Verbatim}          
In order to have more fun, let us define a second dictionary.
\begin{Verbatim}[commandchars=\\\{\}]
{\color{incolor}In [{\color{incolor}9}]:} \PY{n}{Number2Hebrew} \PY{o}{=} \PY{p}{\PYZob{}} \PY{l+m+mi}{1}\PY{p}{:}\PY{l+s+s2}{\PYZdq{}}\PY{l+s+s2}{echad}\PY{l+s+s2}{\PYZdq{}}\PY{p}{,} \PY{l+m+mi}{2}\PY{p}{:}\PY{l+s+s2}{\PYZdq{}}\PY{l+s+s2}{shtaim}\PY{l+s+s2}{\PYZdq{}}\PY{p}{,} \PY{l+m+mi}{3}\PY{p}{:}\PY{l+s+s2}{\PYZdq{}}\PY{l+s+s2}{shalosh}\PY{l+s+s2}{\PYZdq{}}\PY{p}{,} \PY{l+m+mi}{4}\PY{p}{:}\PY{l+s+s2}{\PYZdq{}}\PY{l+s+s2}{arba}\PY{l+s+s2}{\PYZdq{}}\PY{p}{,} \PY{l+m+mi}{5}\PY{p}{:}\PY{l+s+s2}{\PYZdq{}}\PY{l+s+s2}{hamesh}\PY{l+s+s2}{\PYZdq{}}\PY{p}{,} 
                          \PY{l+m+mi}{6}\PY{p}{:}\PY{l+s+s2}{\PYZdq{}}\PY{l+s+s2}{shesh}\PY{l+s+s2}{\PYZdq{}}\PY{p}{,} \PY{l+m+mi}{7}\PY{p}{:}\PY{l+s+s2}{\PYZdq{}}\PY{l+s+s2}{sheva}\PY{l+s+s2}{\PYZdq{}}\PY{p}{,} \PY{l+m+mi}{8}\PY{p}{:}\PY{l+s+s2}{\PYZdq{}}\PY{l+s+s2}{shmone}\PY{l+s+s2}{\PYZdq{}}\PY{p}{,} \PY{l+m+mi}{9}\PY{p}{:} \PY{l+s+s2}{\PYZdq{}}\PY{l+s+s2}{tesha}\PY{l+s+s2}{\PYZdq{}}\PY{p}{,} \PY{l+m+mi}{10}\PY{p}{:} \PY{l+s+s2}{\PYZdq{}}\PY{l+s+s2}{eser}\PY{l+s+s2}{\PYZdq{}}
                        \PY{p}{\PYZcb{}}
\end{Verbatim}
\textbf{Disclaimer:} I don't know any Hebrew, I have taken these names from the youtube video at
\\[0.2cm]
\hspace*{1.3cm}
\href{https://www.youtube.com/watch?v=FBd9QdpqUz0}{https://www.youtube.com/watch?v=FBd9QdpqUz0}.
\\[0.2cm]
Dictionaries can be built via comprehension expressions. Let us
demonstrate this be computing the inverse of the dictionary
Number2English:
\begin{Verbatim}[commandchars=\\\{\}]
{\color{incolor}In [{\color{incolor}10}]:} \PY{n}{English2Number} \PY{o}{=} \PY{p}{\PYZob{}} \PY{n}{Number2English}\PY{p}{[}\PY{n}{name}\PY{p}{]}\PY{p}{:}\PY{n}{name} \PY{k}{for} \PY{n}{name} \PY{o+ow}{in} \PY{n}{Number2English} \PY{p}{\PYZcb{}}
\end{Verbatim}

\begin{Verbatim}[commandchars=\\\{\}]
{\color{incolor}In [{\color{incolor}11}]:} \PY{n}{English2Number}
\end{Verbatim}

\begin{Verbatim}[commandchars=\\\{\}]
{\color{outcolor}Out[{\color{outcolor}11}]:} \{'one': 1,
          'two': 2,
          'three': 3,
          'four': 4,
          'five': 5,
          'six': 6,
          'seven': 7,
          'eight': 8,
          'nine': 9,
          'ten': 10\}
\end{Verbatim}
The example above shows that we can iterate over the keys of a
dictionary. Lets use this to build a dictionary that translates the
English names of numbers into their Hebrew equivalents:
\begin{Verbatim}[commandchars=\\\{\}]
{\color{incolor}In [{\color{incolor}12}]:} \PY{n}{English2Hebrew} \PY{o}{=} \PY{p}{\PYZob{}} \PY{n}{name}\PY{p}{:}\PY{n}{Number2Hebrew}\PY{p}{[}\PY{n}{English2Number}\PY{p}{[}\PY{n}{name}\PY{p}{]}\PY{p}{]}
                            \PY{k}{for} \PY{n}{name} \PY{o+ow}{in} \PY{n}{English2Number} \PY{p}{\PYZcb{}
                         }
\end{Verbatim}

\begin{Verbatim}[commandchars=\\\{\}]
{\color{incolor}In [{\color{incolor}13}]:} \PY{n}{English2Hebrew}
\end{Verbatim}

\begin{Verbatim}[commandchars=\\\{\}]
{\color{outcolor}Out[{\color{outcolor}13}]:} \{'one': 'echad',
          'two': 'shtaim',
          'three': 'shalosh',
          'four': 'arba',
          'five': 'hamesh',
          'six': 'shesh',
          'seven': 'sheva',
          'eight': 'shmone',
          'nine': 'tesha',
          'ten': 'eser'\}
\end{Verbatim}
In order to get the number of entries in a dictionary, we can use the
function \texttt{len}.
\begin{Verbatim}[commandchars=\\\{\}]
{\color{incolor}In [{\color{incolor}14}]:} \PY{n+nb}{len}\PY{p}{(}\PY{n}{English2Hebrew}\PY{p}{)}
\end{Verbatim}

\begin{Verbatim}[commandchars=\\\{\}]
{\color{outcolor}Out[{\color{outcolor}14}]:} 10
\end{Verbatim}
If we want to delete an entry from a dictionary, we can use the keyword
\texttt{del} as follows:
\begin{Verbatim}[commandchars=\\\{\}]
{\color{incolor}In [{\color{incolor}15}]:} \PY{k}{del} \PY{n}{Number2English}\PY{p}{[}\PY{l+m+mi}{1}\PY{p}{]}
\end{Verbatim}

\begin{Verbatim}[commandchars=\\\{\}]
{\color{incolor}In [{\color{incolor}16}]:} \PY{n}{Number2English}
\end{Verbatim}

\begin{Verbatim}[commandchars=\\\{\}]
{\color{outcolor}Out[{\color{outcolor}16}]:} \{2: 'two',
          3: 'three',
          4: 'four',
          5: 'five',
          6: 'six',
          7: 'seven',
          8: 'eight',
          9: 'nine',
          10: 'ten'\}
\end{Verbatim}      
It is important to know that only \emph{immutable} objects can serve as keys in
a dictionary. Therefore, number, strings, tuples, or frozensets can be
used as keys, but lists or sets can not be used as keys.

Given a dictionary \(d\), the method \(d.\texttt{items}()\) can be used
to iterate over all key-value pairs stored in the dictionary \(d\).
\begin{Verbatim}[commandchars=\\\{\}]
{\color{incolor}In [{\color{incolor}17}]:} \PY{p}{\PYZob{}} \PY{n}{pair} \PY{k}{for} \PY{n}{pair} \PY{o+ow}{in} \PY{n}{English2Hebrew}\PY{o}{.}\PY{n}{items}\PY{p}{(}\PY{p}{)} \PY{p}{\PYZcb{}}
\end{Verbatim}


\begin{Verbatim}[commandchars=\\\{\}]
{\color{outcolor}Out[{\color{outcolor}17}]:} \{('eight', 'shmone'),
          ('five', 'hamesh'),
          ('four', 'arba'),
          ('nine', 'tesha'),
          ('one', 'echad'),
          ('seven', 'sheva'),
          ('six', 'shesh'),
          ('ten', 'eser'),
          ('three', 'shalosh'),
          ('two', 'shtaim')\}
\end{Verbatim}
This last example shows that the entries in a dictionary are not ordered.  In this respect, dictionaries are
similar to sets.

\section{Other References}
For reasons of time and space, this lecture has just scratched the surface of what is possible with
\textsl{Python}.  If you want to attain a deeper understanding of \textsl{Python}, here are three places that 
I would recommend:
\begin{enumerate}
\item First, there is the official \textsl{Python} tutorial, which is available at
      \\[0.2cm]
      \hspace*{1.3cm}
      \href{https://docs.python.org/3.9/tutorial/index.html}{\texttt{https://docs.python.org/3.9/tutorial/index.html}}.

      Furthermore, there are a number of good books available.  I would like to suggest the following two
      books.  Both of these books should be available electronically in our library:
\item \emph{The Quick Python Book} written by Naomi R.~Ceder \cite{ceder:2018} is up to date and gives a
      concise introduction to \textsl{Python}.  The book assumes that the reader has some prior programming
      experience.  I would assume that most of our students have the necessary background to feel comfortable
      with this book.
\end{enumerate}
Since \textsl{Python} is \textbf{not} the primary objective of these lecture notes, there is no requirement to read
either the \textsl{Python} tutorial or the books mentioned above.  The primary objective of these
lecture notes is to introduce the main ideas of both \blue{propositional logic} and \blue{predicate logic}.
\textsl{Python} is merely used to illustrate the most important notions from set theory and logic.  You should
be able to pick up enough knowledge of \textsl{Python} by closely inspecting the \textsl{Python} programs
discussed in these lecture notes.  


\section{Reflection}
After having completed this chapter, you should be able to answer the following questions.
\begin{enumerate}[(a)]
\item Which \textsl{Python} data types have been introduced in this chapter?
\item What are the different ways to define a set in \textsl{Python}?
\item How can you build lists and sets via iterators? 
\item How can lists be defined in \textsl{Python}?
\item How does \textsl{Python} support \emph{binary relations}?
\item How does \emph{list slicing} and \emph{list indexing} work?
\item What type of control structures are supported in \textsl{Python}?
\item How can we build sets via \emph{comprehensions}?
\item What are dictionaries?
\item How do we invoke mathematical functions in \textsl{Python}?
\item What are the control structures that are supported in \textsl{Python}.
\item Why do we need \texttt{frozenset}s?
\item How can we compute the power set of a set in \textsl{Python}.
\item How does Euklid's algorithm work?  
\end{enumerate}

%%% Local Variables: 
%%% mode: latex
%%% TeX-master: "logic"
%%% End: 

