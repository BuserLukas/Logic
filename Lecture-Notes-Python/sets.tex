\chapter{Naive Set Theory}
The concept of \href{https://en.wikipedia.org/wiki/set_theory}{set theory} has arisen towards the end of the 19th century
from an effort to put mathematics on a solid foundation.  The creation of a solid foundation was considered necessary as
the concept of infinity increasingly worried mathematicians. 

The essential parts of set theory have been defined by \href{https://de.wikipedia.org/wiki/Georg_Cantor}{Georg
  Cantor} (1845 -- 1918). The first definition of the concept of a set was approximately as follows
\cite{cantor:1895}: 

\begin{center}
\colorbox{red}{\framebox{\colorbox{yellow}{
\begin{minipage}{0.85\linewidth}
  A ``set'' is a \blue{well-defined} collection $M$ of certain objects $x$ of our perception or our thinking.
\end{minipage}}}}
\end{center}
\vspace*{0.2cm}

\noindent
Here, the attribute ``\blue{well-defined}'' expresses the fact that for a given quantity $M$ and an object $x$ we have
to be able to decide whether the object $x$ belongs to the set $M$.  If $x$ belongs to $M$, then $x$ is called an
\blue{element} of the set $M$ and we write this as
\\[0.2cm]
\hspace*{1.3cm}
$x \in M$. 
\\[0.2cm]
The symbol ``$\in$'' is therefore used in set theory as a binary predicate symbol.  We use infix notation when
using this symbol, that is we write $x \in M$ instead of ${\in}(x, M)$.  
Slightly abbreviated we can define the notion of a set as follows: 
\\[0.2cm]
\hspace*{1.3cm}
\textsl{A set is a \blue{well-defined} collection of elements}.
\\[0.2cm]
To mathematically understand the concept of a \blue{well-defined collection of elements},
Cantor introduced the so-called \blue{axiom of comprehension}.
We can formalize this axiom as follows:  If $p(x)$ a \blue{property} that
an object $x$ can have, we can define the set $M$ of all objects that have this
property.  Therefore, the set $M$ can be defined as 
\\[0.2cm]
\hspace*{1.3cm} 
$M := \{ x \;|\; p(x) \}$ 
\\[0.2cm]
and we read this definition as ``$M$ is the set of all $x$ such that $p(x)$ holds''.
Here, a property $p(x)$ is just a formula in which the variable $x$ happens to appear.
We illustrate the axiom of comprehension by an example: If $\mathbb{N}$ is
the set of natural numbers, then we can define the set of all even natural numbers
via the property \\[0.2cm]
\hspace*{1.3cm} $p(x) \;:=\; (\exists y\in \mathbb{N}: x = 2 \cdot y)$. \\[0.2cm]
Using this property, the set of even natural numbers can be defined as \\[0.2cm]
\hspace*{1.3cm} $\{ x \;|\; \exists y\in \mathbb{N}: x = 2 \cdot y \}$. 

Unfortunately, the unrestricted use of the axiom of comprehension leads to serious problems.  To give an
example, let us consider the property of a set to \underline{not} contain itself.  Therefore, we define  
\\[0.2cm]
\hspace*{1.3cm}
 $p(x) := \neg(x \in x)$ 
\\[0.2cm]
and further define the set $R$ as follows: 
\\[0.2cm]
\hspace*{1.3cm} 
$R := \{ x \;|\; \neg (x \in x) \}$.  
\\[0.2cm]
Intuitively, we might expect that no set can contain itself.  However, things turn out to be more complicated.
Let us try to check whether the set $R$ contains itself.  We have
\\[0.2cm]
\hspace*{1.3cm}
$
\begin{array}{cl}
                  & R \in R \\[0.2cm] 
  \Leftrightarrow & R \in \bigl\{ x \;|\; \neg (x \in x) \bigr\} \\[0.2cm] 
  \Leftrightarrow & \neg (R \in R).
\end{array}
$
\\[0.2cm]
So we have shown that
\\[0.2cm]
\hspace*{1.3cm}
$R \in R \;\Leftrightarrow\; \neg(R \in R)$
\\[0.2cm]
holds.  Obviously, this is a contradiction.  As a way out, we can only conclude that the expression \\[0.2cm]
\hspace*{1.3cm} $\{ x \mid \neg (x \in x) \}$ \\[0.2cm]
does not define a set.  This shows that the axiom of comprehension is too general:  Not every expression of the form 
\\[0.2cm] 
\hspace*{1.3cm}
$M := \{ x \mid p(x) \}$ 
\\[0.2cm]
defines a set.  The expression
\\[0.2cm]
\hspace*{1.3cm}
$\bigl\{x \mid \neg(x \in x)\bigr\}$
\\[0.2cm]
has been found by the British logician and philosopher 
\href{http://de.wikipedia.org/wiki/Bertrand_Russell}{Bertrand Russell} (1872 -- 1970).  It is known as
\href{http://de.wikipedia.org/wiki/Russellsche_Antinomy}{Russell's Antinomy}. 

In order to avoid paradoxes such as Russell's antinomy, it is necessary to be more careful when sets are
constructed.  In the following, we will present methods to construct sets that are weaker than the 
axiom of comprehension, but, nevertheless, these methods will be sufficient for our purposes.  We will use the
notation underlying the comprehension axiom and write set definitions in the form
\\[0.2cm]
\hspace*{1.3cm}
$M = \{ x \mid p(x) \}$.  
\\[0.2cm]
However, we won't be allowed to use arbitrary formulas $p(x)$ here.  Instead, the formulas we are going to use
for $p(x)$ have to satisfy some restrictions.  These restrictions will prevent the construction of
self-contradictory sets.

\section{Defining Sets by Listing their Elements}
The simplest way to define a set is to list of all of its elements. These elements are enclosed in the
curly braces  ``\texttt{\{}'' and ``\texttt{\}}'' and are separated by commas.
For example, when we define \\[0.2cm]
\hspace*{1.3cm} $M := \{ 1, 2, 3 \}$, \\[0.2cm]
then the set $M$ contains the elements $1$, $2$ and $3$.
Using  the notation of the axiom of comprehension we could write this set as \\[0.2cm]
\hspace*{1.3cm} 
$M = \{ x \mid x = 1 \vee x = 2 \vee x = 3 \}$.
\\[0.2cm]
Another example of a set that can be created by explicitly enumerating its elements
is the set of all lower case Latin letters.  This set is given as
define: \\[0.2cm]
\hspace*{1.3cm} 
$\{\mathtt{a}, \mathtt{b}, \mathtt{c}, \mathtt{d}, \mathtt{e},
 \mathtt{f}, \mathtt{g}, \mathtt{h}, \mathtt{i}, \mathtt{j}, \mathtt{k}, \mathtt{l},
 \mathtt{m}, \mathtt{n}, \mathtt{o}, \mathtt{p}, \mathtt{q}, \mathtt{r}, \mathtt{s},
 \mathtt{t}, \mathtt{u}, \mathtt{v}, \mathtt{w}, \mathtt{x}, \mathtt{y}, \mathtt{z}\}$.
 \\[0.2cm]
Occasionally, we will use \blue{dot notation} to define a set.  Using dot notation, the set of all lower case
elements is written as
\\[0.2cm]
\hspace*{1.3cm}
$\{ \mathtt{a}, \mathtt{b}, \mathtt{c}, \cdots, \mathtt{x}, \mathtt{y}, \mathtt{z}\} $.
\\[0.2cm]
Of course, if we use dot notation the interpretation of the dots ``$\cdots$'' must always be obvious from the
context of the definition. 

As a last example, we consider the \blue{empty set} $\emptyset$, which is defined as
\\[0.2cm]
\hspace*{1.3cm}
$\emptyset := \{\}$.
\\[0.2cm]
Therefore, the empty set does not contain any element at all.  This set plays an important role in set theory.  This
role is similar to the role played by the number $0$ in algebra.

If a set is defined by listing all of its elements, the order in which the
elements are listed is not important.  For example, we have
\\[0.2cm]
\hspace*{1.3cm}
$\{1,2,3\} = \{3,1,2\}$,
\\[0.2cm]
since both sets contain the same elements.


\section{Predefined Infinite Sets of Numbers}
All sets that are defined by explicitly listing their elements can only have finitely many elements.  
In mathematics there are a number of sets that have an \blue{infinite} number of
elements.  One example is the 
\href{http://en.wikipedia.org/wiki/Natural_number}{set of natural numbers}, which is usually denoted by the symbol $\mathbb{N}$.
Unlike some other authors, I regard the number zero as a natural number.  This is consistent with the
\href{https://en.wikipedia.org/wiki/ISO_31-11}{\textsc{Iso}-standard 31-11}.\footnote{
  The \textsc{Iso} standard 31-11 has been replaced by the
  \href{https://en.wikipedia.org/wiki/ISO_80000-2}{\textsc{Iso}-standard 80000-2},
  but the definition of the set $\mathbb{N}$ has not changed.  In the text, I did not cite \textsc{Iso} 80000-2 because 
  the content of this standard is not freely available, at least not legally.
}
Given the concepts discussed so far, the quantity $\mathbb{N}$ cannot be defined.
We must therefore demand the existence of this set as an \blue{axiom}.  More precisely, we postulate that there is a
set $\mathbb{N}$ which has the following three properties:
\begin{enumerate}
\item $0 \in \mathbb{N}$.
\item If we have a number $n$ such that $n \in \mathbb{N}$, then we also have $n+1 \in \mathbb{N}$.
\item The set $\mathbb{N}$ is the smallest set satisfying the first two conditions.
\end{enumerate}
We write \\[0.2cm]
\hspace*{1.3cm} $\mathbb{N} := \{ 0, 1, 2, 3, \cdots \}$. \\[0.2cm]
Along with the set $\mathbb{N}$ of natural numbers we will use the following sets of numbers: 
\begin{enumerate}
\item $\mathbb{N}^*$ is the set of \blue{positive natural numbers}, so we have
      \\[0.2cm]
      \hspace*{1.3cm}
      $\mathbb{N}^* := \{ n \mid n \in \mathbb{N} \wedge n > 0 \}$.
\item $\mathbb{Z}$ is the set of \blue{integers}, we have
      \\[0.2cm]
      \hspace*{1.3cm}
      $\mathbb{Z} \ = \{ 0, 1, -1, 2, -2, 3, -3, \cdots \}$ 

\item $\mathbb{Q}$ is the set of \blue{rational numbers}, we have
      \\[0.2cm]
      \hspace*{1.3cm}
      $\Bigl\{ \ds\frac{p}{q} \Bigm| p \in \mathbb{Z} \wedge q \in \mathbb{N}^* \Bigr\}$.
\item $\mathbb{R}$ is the set of \blue{real numbers}.

      A clean  mathematically definition of the notion of a \href{https://de.wikipedia.org/wiki/Reelle_number}{real number}
      requires a lot of effort and is out of the scope of this lecture.  If you are interested, a detailed
      description of the construction of real numbers is given in my lecture notes on 
      \href{https://github.com/karlstroetmann/Analysis/blob/master/Skript/analysis.pdf}{Analysis}.
\end{enumerate}

\section{The Axiom of Specification}
The \blue{axiom of specification}, also known as the 
\href{https://en.wikipedia.org/wiki/Axiom_schema_of_specification}{axiom of restricted comprehension},
is a weakening of the comprehension axiom.  The idea behind the axiom of specification
is to use a property $p$ \blue{to select from an existing set $M$ a subset $N$ of those elements
that have the property $p(x)$}: 
\\[0.2cm]
\hspace*{1.3cm}
$N := \{ x\in M \;|\; p(x) \}$ 
\\[0.2cm]
In the notation of the axiom of comprehension this set is written as 
\\[0.2cm]
\hspace*{1.3cm}
$N := \{ x \mid x \in M \wedge p(x) \}$. 
\\[0.2cm]
This is a \blue{restricted} form of the axiom of comprehension, because the condition ``$p(x)$'' that was used in the
axiom of comprehension is now strengthened to the condition ``$x \in M \wedge p(x)$''.


\exampleEng
Using the axiom of restricted comprehension, the set of even natural numbers can be defined as 
\\[0.2cm]
\hspace*{1.3cm}
 $\{ x \in \mathbb{N} \;|\; \exists y\in \mathbb{N}: x = 2 \cdot y \}$. 


\section{Power Sets}
In order to introduce the notion of a \blue{power set} we first have to define the notion of a \blue{subset}.
If $M$ and $N$ are sets, then $M$ is a \blue{subset} of $N$ if and only if each element of the
set $M$ is also an element of the set $N$.  In that case, we write $M \subseteq N$.  Formally, we define
 \\[0.2cm]
\hspace*{1.3cm}
$M \subseteq N \;\stackrel{\mathrm{def}}{\Longleftrightarrow}\; \forall x: (x \in M \rightarrow x \in N)$.

\exampleEng
We have
\\[0.2cm]
\hspace*{1.3cm}
$\{ 1, 3, 5\} \subseteq \{ 1, 2, 3, 4, 5 \}$.
\\[0.2cm]
Furthermore, for any set $M$ we have that
\\[0.2cm]
\hspace*{1.3cm}
$\emptyset \subseteq M$. \eox
\vspace*{0.2cm}

The  \blue{power set} of a set $M$ is now defined as the set of all subsets
of $M$.  We write $2^M$ for the power set of $M$.  Therefore we have
\\[0.2cm] 
\hspace*{1.3cm}
$2^M := \{ x \;|\; x \subseteq M \}$.

\exampleEng
Let us compute the power set of the set $\{1,2,3\}$.  We have \\[0.2cm]
\hspace*{1.3cm}
 $2^{\{1,2,3\}} = \bigl\{ \{\},\, \{1\}, \, \{2\},\, \{3\},\, \{1,2\}, \, \{1,3\}, \{2,3\}, \,\{1,2,3\}\bigr\}$. 
\\[0.2cm]
This set has $8 = 2^3$ elements.  
\eox

In general, if the set $M$ has $m$ different elements, then it can be shown
that the power set $2^M$ has $2^m$ different elements.
More formally, let us designate the number of elements of a finite set $M$ as 
$\textsl{card}(M)$.  Then we have
\\[0.2cm]
\hspace*{1.3cm}
$\textsl{card}\left(2^M\right) = 2^{\textsl{\scriptsize card}(M)}$.
\\[0.2cm]
This explains the notation $2^M$ to denote the power set of $M$.  

\section{The Union of Sets}
If two sets $M$ and $N$ are given, the union
of $M$ and $N$ is the set of all elements that are either in the set $M$ or in the set $N$ or in both $M$ and
in $N$.  This set is written as $M \cup N$.
Formally, this set is defined as 
\\[0.2cm]
\hspace*{1.3cm} $M \cup N := \{ x \;|\; x \in M \vee x \in N \}$. 

\exampleEng
If $M = \{1,2,3\}$ and $N = \{2,5\}$, we have 
\\[0.2cm]
\hspace*{1.3cm} 
$\{1,2,3\} \cup \{2,5\} = \{1,2,3,5\}$.  \eox
\vspace*{0.2cm}

The concept of the union of two sets can be generalized.  Consider
a set $X$ such that the elements of $X$ are sets themselves. For example, the
\blue{power set} of a set $M$ is a set whose elements are sets themselves.  We can form the union of all the 
sets that are elements of the set $X$.  We write this set as $\bigcup X$.  Formally,
we have
\\[0.2cm]
\hspace*{1.3cm} $\bigcup X := \{ y \;|\; \exists x \in X: y \in x \}$.

\exampleEng
If we have \\[0.2cm]
\hspace*{1.3cm}
 $X = \big\{\, \{\},\, \{1,2\}, \, \{1,3,5\}, \, \{7,4\}\,\big\}$, \\[0.2cm]
then \\[0.2cm] 
\hspace*{1.3cm}
 $\bigcup X = \{ 1, 2, 3, 4, 5, 7 \}$. \eox
\vspace*{0.2cm}

\exerciseEng
Assume that $M$ is a subset of $\mathbb{N}$.  Compute the set $\bigcup 2^M$.
\eox

\section{The Intersection of Sets}
If two sets $M$ and $N$ are given, we define the \blue{intersection} of $M$ and $N$ as a set of all objects that are
elements of both $M$ and  $N$.  We write that set as the average $M \cap N$.
Formally, we define 
\\[0.2cm]
\hspace*{1.3cm} $M \cap N := \{ x \mid x \in M \wedge x \in N \}$.

\exampleEng
We calculate the intersection of the sets $M = \{ 1, 3, 5 \}$ and $N = \{ 2, 3, 5, 6 \}$.  We have
\\[0.2cm]
\hspace*{1.3cm} $M \cap N = \{ 3, 5 \}$.
\eox
\vspace*{0.2cm}

The concept of the intersection of two sets can be generalized.  Consider
a set $X$ such that the elements of $X$ are sets themselves. 
We can form the intersection of all the 
sets that are elements of the set $X$.  We write this set as $\bigcap X$.  Formally,
we have
\\[0.2cm]
\hspace*{1.3cm} 
$\bigcap X := \{ y \;|\; \forall x \in X: y \in x \}$.
\vspace*{0.2cm}

\exerciseEng
Assume that $M$ is a subset of $\mathbb{N}$.  Compute the set $\bigcap 2^M$.
\eox

\section{The Difference of Sets}
If $M$ and $N$ are sets, we define the \blue{difference}
of $M$ and $N$ as the set of all objects from $M$ that are not elements of $N$.  The difference of the sets $M$
and $N$ is written as $M\backslash N$ and is formally defined as
 \\[0.2cm]
\hspace*{1.3cm} $M \backslash N := \{ x \mid x \in M \wedge x \not\in N \}$.

\exampleEng
We compute the difference of the sets $M = \{ 1, 3, 5, 7 \}$ and $N = \{ 2, 3, 5, 6 \}$.  We have
\\[0.2cm]
\hspace*{1.3cm} $M \backslash N = \{ 1, 7 \}$. \eox

\section{Image Sets}
If $M$ is a set and $f$ is a function defined for all $x$ of $M$, then the \blue{image of $M$ under $f$}
is defined as follows:
\\[0.2cm]
\hspace*{1.3cm}
$f(M) := \{ y \;|\; \exists x \in M: y = f(x) \}$. 
\\[0.2cm]
This set is also written as
\\[0.2cm]
\hspace*{1.3cm}
$f(M) := \bigl\{ f(x) \;|\; x \in M \}$. 

\exampleEng
The set $Q$ of all square numbers can be defined as 
\\[0.2cm]
\hspace*{1.3cm}
$Q := \{ y \mid \exists x \in \mathbb{N}: y = x^2\}$.
\\[0.2cm]
Alternatively, we can define this set as
\\[0.2cm]
\hspace*{1.3cm}
$Q := \bigl\{ x^2 \mid x \in \mathbb{N} \bigr\}$.
\eox

\section{Cartesian Products}
In order to be able to present the notion of a \href{https://en.wikipedia.org/wiki/Cartesian_product}{Cartesian product},
we first have to introduce the notion of an \href{https://en.wikipedia.org/wiki/Ordered_pair}{ordered pair} of two objects
$x$ and $y$.  The \blue{ordered pair} of $x$ and $y$ is written as
\\[0.2cm] 
\hspace*{1.3cm}
$\langle x, y \rangle$.
\\[0.2cm]
In the literature, the ordered pair of $x$ and $y$ is sometimes written as $(x,y)$, but I prefer the notation
with angle brackets.  The  \blue{first component} of the pair $\langle x, y \rangle$ is $x$, while $y$ is
\blue{the second component}.  Two  ordered pairs $\langle x_1, y_1 \rangle$ and $\langle x_2, y_2 \rangle$ are
equal if and only if they have the same first and second component, i.e.~we have
\\[0.2cm]
\hspace*{1.3cm}
$\langle x_1, y_1 \rangle \,=\,\langle x_2, y_2 \rangle  \;\Leftrightarrow\; x_1 = x_2 \wedge y_1 = y_2$. 
\\[0.2cm]
The \blue{Cartesian product} of two sets $M$ and $N$ is now defined as the set of all ordered pairs such
that the first component is an element of  $M$ and the second component is an element of $N$.
Formally, we define the cartesian product $M \times N$ of the sets $M$ and $N$ as follows:  
\\[0.2cm]
\hspace*{1.3cm}
$M \times N := \big\{ z \mid \exists x\colon \exists y\colon\bigl(z = \langle x,y\rangle \wedge x\in M \wedge y \in N\bigr) \bigr\}$. 
\\[0.2cm]
To be more concise we usually write this as
\\[0.2cm]
\hspace*{1.3cm}
$M \times N := \big\{ \langle x,y\rangle \mid  x\in M \wedge y \in N \}$.

\exampleEng
If $M = \{ 1, 2, 3 \}$ and $N = \{ 5, 7 \}$ we have
\\[0.2cm]
\hspace*{1.3cm} 
$M \times N = \bigl\{ \pair(1,5),\pair(2,5),\pair(3,5),\pair(1,7),\pair(2,7),\pair(3,7)\bigr\}$.
\eox
\vspace*{0.2cm}

\noindent
The notion of an ordered pair can be generalized to the notion of an
\blue{$n$-tuple} where $n$ is a natural number: An $n$-tuple has the form
\\[0.2cm]
\hspace*{1.3cm} $\langle x_1, x_2, \cdots, x_n \rangle$. 
\\[0.2cm]
In a similar way, we can generalize the notion of a Cartesian product of two sets to the Cartesian product of
$n$ sets.  The \blue{general Cartesian product} of $n$ sets  $M_1$, $\cdots$, $M_n$ is defined as follows: \\[0.2cm]
\hspace*{1.3cm}
$M_1 \times \cdots \times M_n =
  \big\{ \langle x_1,x_2,\cdots,x_n \rangle \bigm| x_1\in M_1 \wedge \cdots \wedge x_n \in M_n \big\}
$. 
\\[0.2cm]
Sometimes,  $n$-tuples are called lists.  In this case they are written with the square brackets ``\texttt{[}''
and ``\texttt{]}'' instead of the angle brackets ``$\langle$'' and ``$\rangle$'' that we are using.  

\exerciseEng
Assume that $M$ and $N$ are finite sets.  How can the expression $\textsl{card}(M \times N)$ be reduced to an
expression containing the expressions $\textsl{card}(M)$ and $\textsl{card}(N)$?
\eox

\section{Equality of Sets}
We have now presented all the methods that we will use in this lecture in order to construct sets.
Next, we discuss the notion of \blue{equality} of two sets.  As a set is solely defined by its members,
the question of the equality of two sets is governed by the 
\href{https://en.wikipedia.org/wiki/Axiom_of_extensionality}{axiom of extensionality}:
\vspace*{0.2cm}

\begin{center}     
\colorbox{red}{\framebox{\colorbox{yellow}{ 
\begin{minipage}{0.57\linewidth}
{\sl Two sets are equal if and only if they have the same elements. }  
\end{minipage}}}}
\end{center}
\vspace{0.2cm}

\noindent
Mathematically, we can capture the axiom of extensionality through the formula
\\[0.2cm]
\hspace*{1.3cm} $M = N \;\leftrightarrow\; \forall x: (x \in M \leftrightarrow x \in N)$ 
\\[0.2cm]
An important consequence of this axiom is the fact that the order in which the
elements are listed in a set does not matter.  For example, we have 
\\[0.2cm] 
\hspace*{1.3cm} $\{1,2,3\} = \{3,2,1\}$, 
\\[0.2cm]
because both sets contain the same elements.  Similarly, we have
\\[0.2cm]
\hspace*{1.3cm}
$\{1,2,2,3\} = \{1,1,2,3,3\}$,
\\[0.2cm]
because both these sets contain the elements $1$, $2$, and $3$.  It does not matter how often we list these
elements when defining a set:  An object $x$ either is or is not an element of a given set $M$.  It does not
make sense to say something like ``$M$ contains the object $x$ $n$ times''.\footnote{In the literature, you will find
the concept of a \href{https://en.wikipedia.org/wiki/Multiset}{multiset}.  A \blue{multiset} does not abstract
from the number of occurrences of its elements.  In this lecture, we will not use multisets.}

If two sets are defined by explicitly enumerating their elements, the question whether
these sets are equal is trivial to decide.  However, if a set is defined using the axiom of specification, then
it can be very difficult to decide whether this set is equal to another set.  For 
example, it has been shown that \\[0.2cm]
\hspace*{1.3cm} 
$\{ n \in \mathbb{N}^* \mid \exists x, y, z\in\mathbb{N}^*: x^n + y^n = z^n \} = \{1, 2\}$. 
\\[0.2cm]
However, the proof of this equation is very difficult because this equation
is equivalent to \href{https://en.wikipedia.org/wiki/Fermat%27s_Last_Theorem}{Fermat's conjecture}. 
This conjecture was formulated in 1637 by \href{https://de.wikipedia.org/wiki/Pierre_de_Fermat}{Pierre de Fermat}.  
It took mathematicians more than three centuries to come up with a rigorous proof that validates this conjecture:
In 1994 \href{https://de.wikipedia.org/wiki/Andrew_Wiles}{Andrew Wiles}
and \href{https://de.wikipedia.org/wiki/Richard_Taylor_(Mathematician)}{Richard Taylor} were able to do this.
There are some similar conjectures concerning the equality of sets that are still open mathematical problems. 


\section{Chapter Review}
\begin{enumerate}
\item What is a set?
\item How is the axiom of comprehension defined?  Why can't we use this axiom to define sets?
\item What is the axiom of restricted comprehension?
\item Lists all the methods that have been introduced to define sets.
\item What is the axiom of extensionality?
\end{enumerate}
If you want to develop a deeper understand of set theory, I can highly recommend the book
\emph{Set Theory and Related Topics} by Seymour Lipschutz \cite{lipschutz:1998}.

%%% Local Variables: 
%%% mode: latex
%%% TeX-master: "logic"
%%% End: 
