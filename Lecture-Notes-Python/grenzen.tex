\chapter{Grenzen der Berechenbarkeit}
In jeder Disziplin der Wissenschaft wird die Frage gestellt, welche \blue{Grenzen} die
verwendeten Methoden haben.   Wir wollen daher in diesem Kapitel beispielhaft ein Problem
untersuchen, bei dem die Informatik an ihre Grenzen stößt.  Es handelt sich um das
\href{http://de.wikipedia.org/wiki/Halteproblem}{Halte-Problem}.  

\section{Das Halte-Problem}
Das \blue{Halte-Problem} ist die Frage, ob eine gegebene Funktion $f$ für eine bestimmte Eingabe $x$ 
\blue{terminiert}, ob also der Aufruf $f(x)$ ein Ergebnis liefert oder sich in eine Endlos-Schleife
verabschiedet.   Bevor wir formal beweisen, dass das Halte-Problem im Allgemeinen \blue{unlösbar} ist, wollen 
wir versuchen, anschaulich zu verstehen, warum dieses Problem schwer sein muss.  Dieser informalen
Betrachtung des Halte-Problems ist der nächste Abschnitt gewidmet.  Im Anschluss an diesen Abschluss
beweisen wir dann formal die \blue{Unlösbarkeit} des Halte-Problems. 

\subsection{Informale Betrachtungen zum Halte-Problem}


\begin{figure}[!ht]
\centering
\begin{Verbatim}[ frame         = lines, 
                  framesep      = 0.3cm, 
                  firstnumber   = 1,
                  labelposition = bottomline,
                  numbers       = left,
                  numbersep     = -0.2cm,
                  xleftmargin   = 0.8cm,
                  xrightmargin  = 0.8cm
                ]
    def legendre(n):
        k = n * n + 1;
        while k < (n + 1) ** 2:
            if is_prime(k):
                return True
            k += 1
        return False
    
    def is_prime(k):
        return divisors(k) == {1, k}    
    
    def divisors(k):
        return { t for t in range(1, k+1) if k % t == 0 }
    
    def find_counter_example(n):
        while True:
           if legendre(n):
               n = n + 1
           else:
               print(f'Eureka! No prime between {n}**2 and {n+1}**2!')
               return
\end{Verbatim} 
\vspace*{-0.3cm}
\caption{Eine Funktion zur Überprüfung der von Legendre aufgestellten Vermutung.}
\label{fig:legendre.stlx}
\end{figure}

Um zu verstehen, warum das Halte-Problem schwer ist, betrachten wir  
das in Abbildung \ref{fig:legendre.stlx} gezeigte Programm. 
Dieses Programm ist dazu gedacht, die
\href{http://de.wikipedia.org/wiki/Legendresche_Vermutung}{\emph{Legendresche Vermutung}} zu
überprüfen.  Der französische 
Mathematiker  \href{http://de.wikipedia.org/wiki/Adrien-Marie_Legendre}{Adrien-Marie Legendre} 
(1752 --- 1833) hatte vor etwa 200 Jahren die Vermutung 
aufgestellt, dass zwischen zwei Quadrat-Zahlen immer eine Primzahl liegt.  Die Frage, ob diese
Vermutung richtig ist, ist auch heute noch unbeantwortet.  Die in Abbildung \ref{fig:legendre.stlx}
definierte Funktion $\texttt{legendre}(n)$ überprüft für eine gegebene positive natürliche Zahl $n$,
ob zwischen $n^2$ und $(n+1)^2$ eine Primzahl liegt.  Falls dies, wie von Legendre vermutet, der
Fall ist, gibt die Funktion als Ergebnis \texttt{True} zurück, andernfalls wird \texttt{False}
zurück gegeben.  Die Funktion \texttt{legendre} ist mit Hilfe der Funktion \texttt{is\_prime} definiert.  Für
eine natürliche Zahl $k$ liefert $\texttt{is\_prime}(k)$ genau dann den Wert \texttt{True} zurück, wenn $k$
eine Primzahl ist.  Dazu überprüft die Funktion $\texttt{is\_prime}(k)$ ob die Menge der Teiler von $k$ mit der Menge
die Menge $\{1, k\}$ übereinstimmt.  Die Menge der Teiler wird mit Hilfe der Funktion \texttt{divisors} berechnet.

Abbildung \ref{fig:legendre.stlx} enthält darüber hinaus die Definition der Funktion
$\texttt{findCounterExample}(n)$, die versucht, für eine gegebene positive natürliche Zahl $n$ eine
Zahl $k \geq n$ zu finden, so dass zwischen $k^2$ und $(k+1)^2$ keine Primzahl liegt.  Die Idee bei
der Implementierung dieser Funktion ist einfach:  Zunächst überprüfen wir durch den Aufruf
$\texttt{legendre}(n)$, ob zwischen $n^2$ und $(n+1)^2$
eine Primzahl liegt.  Falls dies der Fall ist, untersuchen wir anschließend das Intervall von
$(n+1)^2$ bis $(n+2)^2$, dann das Intervall von 
$(n+2)^2$ bis $(n+3)^2$ und so weiter, bis wir schließlich eine Zahl $m$ finden, so dass zwischen
$m^2$ und $(m+1)^2$ keine Primzahl liegt.  Falls Legendre Recht hatte, werden wir nie ein solches
$k$ finden und in diesem Fall wird der Aufruf $\texttt{findCounterExample}(2)$ nicht terminieren. 

Nehmen wir nun an, wir hätten ein schlaues Programm, nennen wir es \texttt{stops}, das als Eingabe
eine \textsl{Python} Funktion $f$ und ein Argument $a$ verarbeitet und das uns die Frage, ob die Berechnung von $f(a)$
terminiert, beantworten kann.  Die Idee wäre, dass die Funktion \texttt{stops} die folgende Spezifikation erfüllt:
\\[0.2cm]
\hspace*{1.3cm}
$\texttt{stops}(f, a) = \texttt{true}$ \quad g.d.w. \quad der Aufruf $f(a)$ terminiert.
\\[0.2cm]
Falls der Aufruf $f(a)$ nicht terminiert,  sollte stattdessen $\texttt{stops}(f,a) = \texttt{false}$ gelten.
Wenn wir eine solche Funktion \texttt{stops} hätten, dann könnten wir 
\\[0.2cm]
\hspace*{1.3cm}
\texttt{stops(findCounterExample, 1)}
\\[0.2cm]
aufrufen und wüssten anschließend, ob die Vermutung von Legendre wahr ist oder nicht:  Wenn
\\[0.2cm]
\hspace*{1.3cm}
$\texttt{stops(findCounterExample, 1)} = \texttt{true}$
 \\[0.2cm]
ist, dann würde das heißen,
dass der Funktions-Aufruf $\texttt{findCounterExample}(1)$ terminiert.  Das passiert aber nur dann,
wenn ein Gegenbeispiel gefunden wird.  Würde der Aufruf 
\\[0.2cm]
\hspace*{1.3cm}
\texttt{stops(findCounterExample, 1)}
\\[0.2cm]
stattdessen den Wert  $\texttt{false}$ zurück liefern, so könnten wir schließen, dass der Aufruf \texttt{findCounterExample(1)}
nicht terminiert. Mithin würde die Funktion \texttt{findCounterExample} kein Gegenbeispiel finden und
damit wäre klar, dass die von Legendre aufgestellte Vermutung wahr ist.

Es gibt eine Reihe weiterer offener  mathematischer Probleme, die alle auf die Frage abgebildet
werden können, ob eine gegebene Funktion terminiert.  Daher zeigen die vorhergehenden Überlegungen,
dass es sehr nützlich wäre, eine Funktion wie \texttt{stops} zur Verfügung zu haben.  Andererseits
können wir an dieser Stelle schon ahnen, dass die Implementierung der Funktion \texttt{stops}
nicht ganz einfach sein kann.  

 
\subsection{Formale Analyse des Halte-Problems}
Wir werden in diesem Abschnitt beweisen, dass das Halte-Problem unlösbar ist.  Dazu führen
wir den Begriff einer \blue{Test-Funktion} ein.  
\pagebreak

\begin{Definition}[Test-Funktion] 
Ein String $t$ ist genau dann eine \blue{Test-Funktion} mit Namen $f$, wenn $t$ 
 die Form \\[0.3cm]
\hspace*{1.3cm} \texttt{\symbol{34}\symbol{34}\symbol{34}}         \\
\hspace*{1.3cm} \texttt{def $f$(x):} \\
\hspace*{1.8cm} \textsl{body}        \\
\hspace*{1.3cm} \texttt{\symbol{34}\symbol{34}\symbol{34}}         \\[0.2cm]
hat und sich als \textsl{Python}-Funktion parsen lässt.  Die Variable \textsl{body} steht hier für den Rumpf
der Test-Funktion.  Die Menge der Test-Funktionen bezeichnen wir mit $T\!F$.  \eox
\end{Definition}

\noindent
\textbf{Beispiele}:  
\begin{enumerate}
\item $s_1$ := 
     \texttt{\symbol{34}\symbol{34}\symbol{34}}        \\
     \hspace*{0.8cm} \texttt{def zero(x):} \\
     \hspace*{1.65cm} \texttt{return 0}        \\
     \hspace*{0.8cm} \texttt{\symbol{34}\symbol{34}\symbol{34}}        \\[0.2cm]
      $s_1$ ist eine (sehr einfache) Test-Funktion mit dem Name \texttt{zero}.
\item $s_2$ := 
     \texttt{\symbol{34}\symbol{34}\symbol{34}}        \\
     \hspace*{0.8cm}  \texttt{def loop(x):} \\
     \hspace*{1.65cm} \texttt{while True:}        \\
     \hspace*{2.5cm}  \texttt{x = x + 1}        \\
     \hspace*{0.8cm}  \texttt{\symbol{34}\symbol{34}\symbol{34}}        \\[0.2cm]
      $s_2$ ist eine Test-Funktion mit dem Name \texttt{loop}.
\item $s_3$ := 
     \texttt{\symbol{34}\symbol{34}\symbol{34}}        \\
     \hspace*{0.8cm}  \texttt{def buggy(x):} \\
     \hspace*{1.65cm} \texttt{while True:}        \\
     \hspace*{2.5cm}  \texttt{++x}        \\
     \hspace*{0.8cm}  \texttt{\symbol{34}\symbol{34}\symbol{34}}        \\[0.2cm]
      $s_3$ ist keine Test-Funktion, denn da \textsl{Python} den Präfix-Operator
      ``\texttt{++}'' nicht unterstützt, lässt sich der String $s_3$  nicht fehlerfrei als \textsl{Python} 
      parsen.
\item $s_4$ := 
     \texttt{\symbol{34}\symbol{34}\symbol{34}}        \\
     \hspace*{0.8cm}  \texttt{def sum(x, y):} \\
     \hspace*{1.65cm} \texttt{while True:}        \\
     \hspace*{2.5cm}  \texttt{return x + y}        \\
     \hspace*{0.8cm}  \texttt{\symbol{34}\symbol{34}\symbol{34}}        \\[0.2cm]
      $s_4$ ist keine Test-Funktion, denn ein String ist nur dann eine Test-Funktion, wenn die durch den String
      definierte Funktion mit \blue{genau einen} Parameter aufgerufen wird.
\end{enumerate}
Um das Halte-Problem übersichtlicher formulieren zu können, führen wir noch vier
zusätzliche Notationen ein.
\begin{Notation}[\textsl{name}, $\leadsto$, $\downarrow$, $\uparrow$]
{\em
Ist $t$ eine Testfunktion mit Namen $f$, so schreiben wir 
\\[0.2cm]
\hspace*{1.3cm}
$\textsl{name}(t) = f$.
\\[0.2cm]
Ist $f$ der Name einer \textsl{Python}-Funktion, die $k$ Argumente verarbeitet und sind $a_1$, $\cdots$, $a_k$ mögliche
Argumente, mit denen wir diese Funktion aufrufen können,
so schreiben wir \\[0.3cm]
\hspace*{1.3cm} $f(a_1, \cdots, a_k) \leadsto r$ \\[0.3cm]
wenn der Aufruf $f(a_1, \cdots, a_k)$ das Ergebnis $r$ liefert.  Sind wir an dem Ergebnis
selbst nicht interessiert, sondern wollen nur angeben, dass ein Ergebnis existiert, so
schreiben wir \\[0.3cm]
\hspace*{1.3cm} $f(a_1, \cdots, a_k) \,\downarrow$ \\[0.3cm]
und sagen, dass der Aufruf $f(a_1, \cdots, a_k)$ \blue{terminiert}.
Terminiert der Aufruf $f(a_1, \cdots, a_k)$ nicht, so schreiben wir \\[0.3cm]
\hspace*{1.3cm} $f(a_1, \cdots, a_k) \,\uparrow$ \\[0.3cm]
und sagen, dass der Aufruf $f(a_1, \cdots, a_k)$ \blue{divergiert}.  Diese Notation
verwenden wir auch, wenn der Aufruf  $f(a_1, \cdots, a_k)$ mit einer Fehlermeldung abbricht.
\hspace*{\fill} $\Box$
}
\end{Notation}

\noindent
\textbf{Beispiele}: Legen wir die Funktions-Definitionen zugrunde, die wir im Anschluss an
die Definition des Begriffs der Test-Funktion gegeben haben, so gilt:
\begin{enumerate}
\item {\tt zero(1)} $\leadsto 0$
\item {\tt zero(1)} $\downarrow$
\item {\tt loop(0)} $\uparrow$
\end{enumerate} 

\noindent
Das \blue{Halte-Problem} für
\textsl{Python}-Funktionen ist die Frage, ob es eine \textsl{Python}-Funktion \\[0.2cm] 
\hspace*{1.3cm} \texttt{def stops($t$, $a$):}\\
\hspace*{2.1cm} \textsl{body}
\\[0.2cm]
gibt, die als Eingaben eine Testfunktion $t$ und einen String $a$ erhält und die folgende
Eigenschaft hat:
\begin{enumerate}
\item $t \not\in T\!F \quad\Leftrightarrow\quad \texttt{stops}(t, a) \leadsto 2$.

      Der Aufruf \texttt{stops($t$, $a$)} liefert genau dann den Wert 2 zurück, 
      wenn $t$ keine Test-Funktion ist.
\item $t \in T\!F \,\wedge\, \textsl{name}(t) = f \,\wedge\, f(a)\downarrow \quad\Leftrightarrow\quad
       \texttt{stops}(t, a) \leadsto 1$.

      Der Aufruf \texttt{stops($t$, $a$)} liefert genau dann den Wert 1 zurück,
      wenn $t$ eine Test-Funktion mit Namen $f$ ist und der Aufruf $f(a)$ terminiert.

\item $t \in T\!F  \,\wedge\, \textsl{name}(t) = f \,\wedge\, t(a)\uparrow \quad\Leftrightarrow\quad
       \texttt{stops}(t, a) \leadsto 0$.

      Der Aufruf \texttt{stops($t$, $a$)} liefert genau dann den Wert 0 zurück,
      wenn $t$ eine Test-Funktion ist und der Aufruf $t(a)$ \underline{nicht} terminiert.
\end{enumerate}
Falls eine \textsl{Python}-Funktion \texttt{stops} mit den obigen Eigenschaften existiert, dann
sagen wir, dass das Halte-Problem für \textsl{Python} entscheidbar ist.

\begin{Theorem}[\href{http://de.wikipedia.org/wiki/Alan_Turing}{Alan Turing}, 1936]
{\em
  Das Halte-Problem ist unentscheidbar.
} 
\end{Theorem}

\noindent
\textbf{Beweis}:  Zunächst eine Vorbemerkung.  Um die Unentscheidbarkeit des
Halte-Problems nachzuweisen, müssen wir zeigen, dass etwas, nämlich eine Funktion mit
gewissen Eigenschaften nicht existiert.  Wie kann so ein Beweis überhaupt funktionieren?
Wie können wir überhaupt zeigen, dass irgendetwas nicht existiert?
Die einzige Möglichkeit zu zeigen, dass etwas nicht existiert ist indirekt:
Wir nehmen also an, dass eine Funktion \texttt{stops} existiert, die das Halte-Problem löst.
Aus dieser Annahme werden wir einen Widerspruch ableiten.  Dieser Widerspruch zeigt
uns dann, dass eine Funktion \texttt{stops} mit den gewünschten Eigenschaften nicht
existieren kann.
Um zu einem Widerspruch zu kommen, definieren wir den String \texttt{turing} wie in Abbildung
\ref{fig:turing-string} gezeigt.

\begin{figure}[!h]
  \centering
\begin{Verbatim}[ frame         = lines, 
                  framesep      = 0.3cm, 
                  labelposition = bottomline,
                  numbers       = left,
                  numbersep     = -0.2cm,
                  xleftmargin   = 0.8cm,
                  xrightmargin  = 0.8cm,
                  commandchars  = \\\{\},
                ]
    turing = """
             def alan(x):
                 result = stops(x, x)
                 if result == 1:
                     while True:
                         print("... looping ...")
                 return result
             """ 
\end{Verbatim}
  \vspace*{-0.3cm}
  \caption{Die Definition des Strings \texttt{turing}.}
  \label{fig:turing-string}
\end{figure}

Mit dieser Definition ist klar, dass \texttt{turing} eine Test-Funktion mit Namen \texttt{alan} ist: \\[0.3cm]
\hspace*{1.3cm} $\texttt{turing} \in T\!F$ \quad und \quad $\textsl{name}(\texttt{turing}) = \texttt{alan}$. 
\\[0.3cm]
Damit sind wir in der Lage, den String \texttt{turing} als Eingabe der Funktion \texttt{stops}
zu verwenden.  Wir betrachten nun den folgenden Aufruf: \\[0.3cm]
\hspace*{1.3cm} \texttt{stops(\texttt{turing}, \texttt{turing});} \\[0.3cm]
Da \texttt{turing} eine Test-Funktion mit dem Namen \texttt{alan} ist und der Aufruf von \texttt{alan} mit dem Argument \texttt{turing}
auch nicht zu einem Fehler führen darf, können nur zwei Fälle auftreten:
\\[0.1cm]
\hspace*{1.3cm} 
$\texttt{stops(\texttt{turing}, \texttt{turing})} \leadsto 0 \quad \vee\quad
 \texttt{stops(\texttt{turing}, \texttt{turing})} \leadsto 1$. \\[0.1cm]
Diese beiden Fälle analysieren wir nun im Detail:
\begin{enumerate}
\item $\texttt{stops(\texttt{turing}, \texttt{turing})} \leadsto 0$. 

      Nach der Spezifikation von \texttt{stops} bedeutet dies \\[0.1cm]
      \hspace*{1.3cm} $\texttt{alan}(\texttt{turing}) \uparrow$ \\[0.1cm]
      Schauen wir nun, was wirklich beim Aufruf \texttt{alan(turing)} passiert:
      In Zeile 2 erhält die Variable \texttt{result} den Wert 0 zugewiesen.  In Zeile 3
      wird dann getestet, ob \texttt{result} den Wert 1 hat.  Dieser Test schlägt fehl.
      Daher wird der Block der \texttt{if}-Anweisung nicht ausgeführt und die Funktion liefert als
      nächstes in Zeile 8 den Wert 0 zurück.  Insbesondere terminiert der Aufruf also, im
      Widerspruch zu dem, was die Funktion \texttt{stops} behauptet hat. $\red{\lightning}$

      Damit ist der erste Fall ausgeschlossen.
\item  $\texttt{stops}(\texttt{turing}, \texttt{turing}) \leadsto 1$. 

      Aus der Spezifikation der Funktion \texttt{stops} folgt, dass der Aufruf
      \texttt{alan(turing)} terminiert: \\[0.1cm]
      \hspace*{1.3cm} $\texttt{alan(turing)} \downarrow$ \\[0.1cm]
      Schauen wir nun, was wirklich beim Aufruf \texttt{alan(turing)} passiert:
      In Zeile 2 erhält die Variable \texttt{result} den Wert 1 zugewiesen.  In Zeile 3
      wird dann getestet, ob \texttt{result} den Wert 1 hat.  Diesmal gelingt der Test.
      Daher wird der Block der \texttt{if}-Anweisung ausgeführt.  Dieser Block
      besteht aber nur aus einer Endlos-Schleife, aus der wir nie wieder zurück kommen.
      Das steht im Widerspruch zu dem, was die Funktion \texttt{stops} behauptet hat.
      $\red{\lightning}$

      Damit ist der zweite Fall ausgeschlossen.
\end{enumerate}
Insgesamt haben wir also in jedem Fall einen Widerspruch erhalten.  
Damit muss die Annahme, dass die \textsl{Python}-Funktion \texttt{stops}
das Halte-Problem löst, falsch sein, denn diese Annahme ist die Ursache für die Widersprüche, die
wir erhalten haben.  Insgesamt haben wir daher gezeigt, dass es keine \textsl{Python}-Funktion
geben kann, die das Halte-Problem löst. \hspace*{\fill} $\Box$
\vspace*{0.3cm}

\noindent
\textbf{Bemerkung}:
Der Nachweis, dass das Halte-Problem unlösbar ist, wurde 1936 von Alan Turing (1912 -- 1954)
\cite{turing:36} erbracht.  Turing hat das Problem damals natürlich nicht für die Sprache
\textsl{Python} gelöst, sondern für die heute nach ihm benannten
\href{https://de.wikipedia.org/wiki/Turingmaschine}{Turing-Maschinen}.   
Eine Turing-Maschine ist abstrakt gesehen nichts anderes als eine Beschreibung eines
Algorithmus.  Turing hat also gezeigt, dass es keinen Algorithmus gibt, der entscheiden
kann, ob ein gegebener anderer Algorithmus terminiert.
\vspace*{0.3cm}

\noindent
\textbf{Bemerkung}:
An dieser Stelle können wir uns fragen, ob es vielleicht eine andere Programmier-Sprache
gibt, in der wir das Halte-Problem dann vielleicht doch lösen könnten.  
Wenn es in dieses Programmier-Sprache Prozeduren, \texttt{if}-Verzweigungen und
\texttt{while}-Schleifen gibt, und wenn wir dort 
Programm-Texte als Argumente von Funktionen übergeben können, dann ist leicht zu sehen,
dass der obige Beweis der 
Unlösbarkeit des Halte-Problems sich durch geeignete syntaktische Modifikationen auch auf
die andere Programmier-Sprache übertragen lässt.


\section{Unlösbarkeit des Äquivalenz-Problems}
Es gibt noch eine ganze Reihe anderer Funktionen, die nicht berechenbar sind.  In der
Regel werden wir den Nachweis, dass eine bestimmt Funktion nicht berechenbar ist, indirekt führen
und annehmen, dass die gesuchte Funktion doch berechenbar ist.  Unter dieser Annahme
konstruieren wir dann eine Funktion, die das Halte-Problem löst, was im Widerspruch zu der Unlösbarkeit
des Halte-Problems steht.
Dieser Widerspruch zwingt uns zu der Folgerung, dass die gesuchte Funktion nicht berechenbar ist.
Wir werden dieses Verfahren an einem Beispiel demonstrieren.  Vorweg benötigen wir aber
noch eine Definition.

\begin{Definition}[partiell äquivalent, $\simeq$]  
Es seien $f_1$ und $f_2$ die Namen zwei \textsl{Python}-Funktionen und
  $a_1$, $\cdots$, $a_k$ seien Argumente, mit denen wir diese Funktionen füttern können. Wir definieren \\[0.1cm]
\hspace*{1.3cm} $t_1(a_1,\cdots,a_k) \simeq t_2(a_1,\cdots,a_k)$ \\[0.1cm]
g.d.w. einer der beiden folgen Fälle auftritt:
\begin{enumerate}
\item $f_1(a_1,\cdots,a_k)\uparrow \quad\wedge\quad f_2(a_1,\cdots,a_k)\uparrow$,

      beide Funktionen divergieren also für die gegebenen Argumente.
\item $\exists r: \Bigl(f_1(a_1,\cdots,a_k) \leadsto r \;\wedge\; f_2(a_1,\cdots,a_k) \leadsto r\Bigr)$,

      die beiden Funktionen liefern also für die gegebenen Argumente das gleiche Ergebnis.
\end{enumerate}
      In diesem Fall sagen wir, dass die beiden Funktions-Aufrufe 
      $f_1(a_1,\cdots,a_k) \simeq f_2(a_1,\cdots,a_k)$ \blue{partiell äquivalent} sind.  
      \qed
\end{Definition}

\noindent
Wir kommen jetzt zum \blue{Äquivalenz-Problem}.  Die Funktion $\texttt{equal}$ sei wie folgt definiert:
\\[0.2cm]
\hspace*{1.3cm} \texttt{def equal(t1, t2, a):} \\
\hspace*{2.2cm} \textsl{body}
\\[0.2cm]
Zusätzlich erfüllt \texttt{equal} die folgende Spezifikation:
\begin{enumerate}
\item $t_1 \not\in T\!F \;\vee\; t_2 \not\in T\!F \quad\Leftrightarrow\quad \texttt{equal}(t_1, t_2, a) \leadsto 2$.
\item Falls 
  \begin{enumerate}
  \item $t_1 \in T\!F$ \quad und \quad $\textsl{name}(t_1) = f_1$,
  \item $t_2 \in T\!F$ \quad und \quad $\textsl{name}(t_2) = f_2$ \quad und
  \item $f_1(a) \simeq f_2(a)$
  \end{enumerate}
    gilt, dann muss gelten: \\[0.1cm]
   \hspace*{1.3cm}  $\texttt{equal}(t_1, t_2, a) \leadsto 1$.
\item Ansonsten gilt \\[0.1cm]
      \hspace*{1.3cm} $\texttt{equal}(t_1, t_2, a) \leadsto 0$.
\end{enumerate}
Wir sagen, dass eine Funktion, die der eben angegebenen Spezifikation genügt, das
\blue{Äquivalenz-Problem} löst.

\begin{Theorem}[Rice, 1953]
Das Äquivalenz-Problem ist unlösbar.  
\end{Theorem}

\noindent
\textbf{Beweis}:
Wir führen den Beweis indirekt und nehmen
an, dass es doch eine Implementierung der Funktion \texttt{equal} gibt, die das
Äquivalenz-Problem löst.  Wir betrachten die in Abbildung 
\ref{fig:stops} angegeben Implementierung der Funktion \texttt{stops}.


\begin{figure}[!h]
  \centering
\begin{Verbatim}[ frame         = lines, 
                  framesep      = 0.3cm, 
                  labelposition = bottomline,
                  numbers       = left,
                  numbersep     = -0.2cm,
                  xleftmargin   = 0.3cm,
                  xrightmargin  = 0.3cm
                ]
     def stops(t, a):
         l = """
             def loop(x): 
                 while True:
                     x = 1
             """ 
         e = equal(l, t, a);
         if e == 2:
             return 2
         else:
             return 1 - e
\end{Verbatim}
  \vspace*{-0.3cm}
  \caption{Eine Implementierung der Funktion \texttt{stops}.}
  \label{fig:stops}
\end{figure}

Zu beachten ist, dass in Zeile 3 die Funktion \texttt{equal} mit einem String aufgerufen
wird, der eine Test-Funktion ist.  Diese 
Test-Funktion hat die
folgende Form:
\begin{verbatim}
            def loop(x): 
                 while True:
                     x = 1
\end{verbatim}
Es ist offensichtlich, dass diese Funktion für keine Eingabe terminiert.
Ist also das Argument $t$ eine Test-Funktion mit Namen $f$, so liefert die Funktion
\texttt{equal} immer dann den Wert 1, 
wenn $f(a)$ nicht terminiert, andernfalls muss sie den Wert 0 zurück geben.
Damit liefert die Funktion \texttt{stops} aber für eine Test-Funktion $t$ mit Namen $f$
und ein Argument $a$ genau dann 1, wenn der Aufruf $f(a)$ terminiert und würde folglich das Halte-Problem
lösen.  Das kann nicht sein, also kann es keine Funktion  \texttt{equal}
geben, die das Äquivalenz-Problem löst. 
\qed
\vspace*{0.3cm}

\noindent
Die Unlösbarkeit des Äquivalenz-Problems und vieler weiterer praktisch interessanter
Probleme folgen aus dem 1953 von Henry G.~Rice \cite{rice:53} bewiesenen
\href{http://de.wikipedia.org/wiki/Satz_von_Rice}{Satz von Rice}.

\section{Reflexion}
Beantworten Sie die folgenden Fragen.
\begin{enumerate}
\item Wie ist das Halteproblem definiert?
\item Versuchen Sie, die Definition der Funktion \texttt{turing} aus dem Gedächtnis aufzuschreiben und führen Sie
      dann den Nachweis, dass das Halteproblem nicht lösbar ist.
\item Wie haben wir das Äquivalenz-Problem definiert?
\item Können Sie den Beweis der Unlösbarkeit des Äquivalenz-Problems wiedergeben?
\end{enumerate}


%%% Local Variables: 
%%% mode: latex
%%% TeX-master: "logic"
%%% ispell-local-dictionary: "deutsch8"
%%% End: 
