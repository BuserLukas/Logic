\chapter{Introduction}
For the uninitiated, \href{https://en.wikipedia.org/wiki/Mathematical_logic}{mathematical logic} 
is both quite abstract and pretty arcane.  
In this short chapter, I would like to motivate the reason that you have to learn
logic in order to become a computer scientist.  After that, I will give a short overview of this
lecture.

\section{Motivation}
Modern software systems are among the most complex systems developed by mankind.  You can get a
sense of the complexity of these systems if you look at the amount of work that is necessary to
build and maintain complex software systems.  Today it is quite common that complex software projects require
more than a thousand collaborating developers to  
develop a new system.  The failure of a project of this size is very costly.
The page
\\[0.2cm]
\hspace*{0.8cm}
\href{http://spectrum.ieee.org/static/the-staggering-impact-of-it-systems-gone-wrong}{Staggering Impact of IT Systems Gone Wrong} 
\\[0.2cm]
\hspace*{0.8cm}
(\texttt{http://spectrum.ieee.org/static/the-staggering-impact-of-it-systems-gone-wrong})
\\[0.2cm]
presents several examples showing big software projects that have failed and have subsequently caused huge
financial losses.  To present just one recent example, the 
\href{https://www.tagesschau.de/inland/it-konsolidierung-bund-101.html}{consolidation of Germany's federal IT system} 
is currently in a crisis:  Whereas the costs had originally been estimated at 1 billion euros, the current
estimate is at 3.42 billion euros and recent estimates suggest that even this number will be exceeded.   
This and numerous other examples show that the development of complex software systems requires a high level
of precision and diligence.  Hence, the development of software needs a solid scientific
foundation.  Both \href{https://en.wikipedia.org/wiki/Mathematical_logic}{mathematical logic} and 
\href{https://en.wikipedia.org/wiki/Set_theory}{set theory}
are important parts of this foundation.  Furthermore, both set theory and logic have immediate applications in
computer science. 
\begin{enumerate}
\item Logic can be used to specify the \blue{interfaces} of complex systems.  
\item The correctness of digital circuits can be verified using \blue{automatic theorem provers} that are based on
      propositional logic.
\item Set theory and the theory of relations is one of the foundations of \blue{relational databases}.
\end{enumerate}
It is easy to extend this enumeration.  However, besides their immediate applications, 
there is another reason you have to study both logic and set theory: Without the proper use of
{\color{blue}abstractions}, complex software systems cannot be managed.  After all, nobody is able to keep
millions of lines of program code in her head.  The only way to construct and manage a software system of this
size is to introduce the right abstractions and to develop the system in layers.  Hence, the ability
to work with abstract concepts is one of the main virtues of a modern computer scientist.  
Exposing students to logic and set theory trains their abilities to grasp abstract concepts.

From my past teaching experience I know that many students think that a good programmer already is a
good computer scientist.  In reality, we have
\\[0.2cm]
\hspace*{1.3cm}
$\textrm{good programmer} \not= \textrm{good computer scientist}$.
\\[0.2cm]
This should not be too surprising.  After all, there is no reason to believe that a good bricklayer is a good
architect and neither is a good architect necessarily a good bricklayer.
In computer science, a good programmer need not be a scientist at all, while a {\color{blue}computer
  \underline{scientist}}, by its very name, is a {\color{blue}scientist}.  
There is no denying that {\color{blue}mathematics} in general and 
{\color{blue}logic} in particular is an important part of science.  Furthermore, these topics form the
foundation of computer science.  Therefore, you should master them.  In addition, this
part of your scientific education is much more permanent than the knowledge of a particular programming
language.  Nobody knows which programming language will be \emph{en vogue} in 10 years from now.  In three 
years, when you start your professional career, quite a lot of you will have to learn a new
programming language.   Then your ability to quickly grasp new concepts will be much more important than your
skills in any particular programming language. 

\section{Overview} 
The first lecture in theoretical computer science creates the foundation that is needed for future lectures.
This lecture deals mostly with mathematical logic and is structured as follows.
\begin{enumerate}[(a)]
\item We begin our lecture with a short introduction of \blue{set theory}.  A basic understanding of set theory is
      necessary for us to formally define the \blue{semantics} of both \blue{propositional logic} and \blue{first order logic}.
\item We proceed to introduce the programming language \textsl{Python}.

      As the concepts introduced in this lecture are quite abstract, it is beneficial to clarify the main
      ideas presented in this lectures via programs.  The programming language
      \href{http://python.org}{\textsl{Python}} supports sets together with the most important operations
      defined on sets.  Therefore it is
      suitable to implement most of the abstract ideas presented in this lecture.  According to the 
      \href{http://ieee.org}{\textsc{Ieee}} (\underline{I}nstitute of \underline{E}lectrical and
      \underline{E}lectronics \underline{E}gineers),  \textsl{Python} is the 
      \href{https://spectrum.ieee.org/static/interactive-the-top-programming-languages-2019}{most popular programming language}.
      Furthermore, \textsl{Python}
      \href{https://cacm.acm.org/blogs/blog-cacm/176450-python-is-now-the-most-popular-introductory-teaching-language-at-top-u-s-universities/fulltext}{is
      the most popular introductory teaching language at top U.S. universities}.   For these reasons I
      have decided to base these lectures on \textsl{Python}.

\item Next, we investigate the limits of computability.

      For certain problems there is no algorithm that can solve the problem algorithmically. 
      For example, the question whether a given program will \blue{terminate} for a given input is not
      \blue{decidable}.  This is known as the \href{https://en.wikipedia.org/wiki/Halting_problem}{halting problem}.  
      We will prove the \blue{undecidability} of the halting problem in the third chapter. 
\item The fourth chapter discusses \href{https://en.wikipedia.org/wiki/Propositional_calculus}{propositional logic}.

      In logic, we distinguish between  \blue{propositional logic},
      \blue{first order logic}, and \blue{higher order logic}.  \blue{Propositional} logic is only
      concerned with the \blue{logical connectives}
      \\[0.2cm]
      \hspace*{1.3cm}
      ``$\neg$'', ``$\wedge$'', ``$\vee$'', ``$\rightarrow$'' und ``$\leftrightarrow$'',
      \\[0.2cm]
      while \blue{first-order logic} also investigates the \blue{quantifiers}
      \\[0.2cm]
      \hspace*{1.3cm}
      ``$\forall$'' and ``$\exists$'',
      \\[0.2cm]
      where these quantifiers range over the objects of the \blue{domain of discourse}.
      Finally, in \blue{higher order logic} the quantifiers also range over \blue{sets}, \blue{functions}, and
      \blue{predicates}. 

      As propositional logic is easier to grasp than first-order logic, we start our investigation
      of logic with propositional logic.  Furthermore, propositional logic has the advantage of
      being \blue{decidable}:  We will present an algorithm that can check whether a propositional formula
      is universally valid.  In contrast to propositional logic, first-order logic is not decidable.

      Next, we discuss applications of propositional logic:  We will show how the \blue{8 queens problem} 
      can be reduced to the question whether a formula from propositional logic is satisfiable.  We present
      the algorithm of \blue{Davis and Putnam} that can decide the satisfiability of a propositional formula.
      This algorithm is therefore able to solve the 8 queens problem.  
\item Finally, we discuss \href{https://en.wikipedia.org/wiki/First-order_logic}{first-order logic}.

      The most important concept of the last chapter will be the notion of a \blue{formal proof} in
      first order logic.  To this end, we introduce a \blue{formal proof system} that is
      \blue{complete} for first order logic.  \blue{Completeness} means that we will develop an
      algorithm that can \blue{prove} the correctness of every first-order formula that is
      universally valid.  This algorithm is the foundation of \blue{automated theorem proving}.

      As an application of theorem proving we discuss the systems \href{https://www.cs.unm.edu/~mccune/mace4/}{Prover9} and
      \href{https://www.cs.unm.edu/~mccune/mace4/}{Mace4}. \blue{Prover9} is an automated theorem prover, while
      \blue{Mace4} can be used to refute a mathematical conjecture.
\end{enumerate}

%%% Local Variables: 
%%% mode: latex
%%% TeX-master: "logic"
%%% End: 
