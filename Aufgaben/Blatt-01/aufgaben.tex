\documentclass{article}
\usepackage{german}
\usepackage[latin1]{inputenc}
\usepackage{a4wide}
\usepackage{amssymb}
\usepackage{fancyvrb}
\usepackage{alltt}
\usepackage{hyperref}
\usepackage{fancyvrb}
\usepackage{fancyhdr}
\usepackage{lastpage} 
\usepackage[all]{hypcap}
\hypersetup{
	colorlinks = true, % comment this to make xdvi work
	linkcolor  = blue,
	citecolor  = red,
        filecolor  = Gold,
        urlcolor   = [rgb]{0.9, 0.1, 0.0},
	pdfborder  = {0 0 0} 
}

\renewcommand*{\familydefault}{\sfdefault}

\newcommand{\blue}[1]{{\color{blue}#1}}
\newcommand{\red}[1]{{\color{red}#1}}

\pagestyle{fancy}
\lhead{Theoretische Informatik \texttt{I}}
\rhead{Aufgaben-Blatt Nr. 1}
\fancyfoot[C]{--- \thepage/\pageref{LastPage}\ ---}

\renewcommand{\labelenumi}{(\alph{enumi})}
\renewcommand{\labelenumii}{\arabic{enumii}.}

\newcounter{aufgabe}
\newcommand{\exercise}{\vspace*{0.3cm}
\stepcounter{aufgabe}

\noindent
\textbf{Aufgabe \arabic{aufgabe}}: }

\newcommand{\club }{\ensuremath{\clubsuit   }}
\newcommand{\spade}{\ensuremath{\spadesuit  }}
\newcommand{\heart}{\ensuremath{\heartsuit  }}
\newcommand{\diamo}{\ensuremath{\diamondsuit}}
\def\pair(#1,#2){\langle #1, #2 \rangle}


\begin{document}


\exercise
Eine Zahl $m\in \mathbb{N}$ ist ein \blue{echter Teiler} einer Zahl
$n \in \mathbb{N}$ genau dann, wenn $m$ ein Teiler von $n$ ist und wenn au�erdem $m < n$ gilt.

Eine Zahl $n \in \mathbb{N}$ hei�t \blue{perfekt}, wenn $n$ gleich der Summe aller echten Teiler von
$n$ ist. Zum Beispiel ist die Zahl $6$ perfekt, denn  die Menge der echten Teiler
von 6 ist $\{1,2,3\}$ und es gilt $1 + 2 + 3 = 6$.
\begin{enumerate}
\item Implementieren Sie eine Prozedur \texttt{echteTeiler}, so dass der Aufruf
      $\texttt{echteTeiler}(n)$ f�r eine nat�rliche Zahl $n$ die Menge aller echten Teiler von
      $n$ berechnet.
\item Berechnen Sie die Menge aller perfekten Zahlen, die kleiner als $10\,000$ sind.
\end{enumerate}
\vspace{0.3cm}

\exercise
\begin{enumerate}
\item Implementieren Sie eine Prozedur \texttt{gt}, so dass der Aufruf $\texttt{gt}(m,n)$ f�r zwei nat�rliche
      Zahlen $m$ und $n$ die Menge aller \blue{gemeinsamen Teiler} von $m$ und $n$ berechnet. 

      \textbf{Hinweis}: Berechnen Sie zun�chst die Menge der Teiler von $m$ und 
      die Menge der Teiler von $n$.  �berlegen Sie, wie die Mengenlehre Ihnen weiterhilft,
      wenn Sie diese beiden Mengen berechnet haben.
\item Implementieren Sie nun eine Prozedur \texttt{ggt}, so dass der Aufruf $\texttt{ggt}(m,n)$ 
      den \href{https://de.wikipedia.org/wiki/Gr��ter_gemeinsamer_Teiler}{gr��ten gemeinsamen Teiler} 
      der beiden Zahlen $m$ und $n$ berechnet. 
\end{enumerate}
\vspace{0.3cm}

\exercise
Implementieren Sie  eine Prozedur \texttt{kgv}, so dass der Aufruf
$\texttt{kgv}(m,n)$ f�r zwei nat�rliche  Zahlen $m$ und $n$ das 
\href{https://de.wikipedia.org/wiki/Kleinstes_gemeinsames_Vielfaches}{kleinste gemeinsame Vielfache}
der Zahlen $m$ und $n$ berechnet. 
 \vspace{0.2cm}

\noindent
\textbf{Hinweis}: Es gilt $\texttt{kgv}(m,n) \leq m \cdot n$.
\vspace{0.3cm}

\exercise
\begin{enumerate}
\item Implementieren Sie eine Funktion \texttt{subsets}, so dass $\texttt{subsets}(M, k)$ f\"ur eine
      Menge $M$ und eine nat\"urliche Zahl $k$ die Menge aller der Teilmengen von $M$ berechnet, die
      genau $k$ Elemente haben.

      \textbf{Hinweis}:  Versuchen Sie, die Funktion  $\texttt{subsets}(M, k)$ rekursiv zu
      implementieren.  
\item Geben Sie eine Implementierung der Funktion \texttt{power} an, bei der Sie die Funktion
      \texttt{subsets} aus Teil (a) dieser Aufgabe verwenden.  F�r eine Menge $M$ soll die Funktion
      $\texttt{power}(M)$ die Potenz-Menge $2^M$ berechnen.
      
\end{enumerate}
\pagebreak

\exercise
Eine Liste der Form $[a, b, c]$ wird als 
\blue{geordnetes} \href{http://de.wikipedia.org/wiki/Pythagoreisches_Tripel}{pythagoreisches Tripel}
bezeichnet, wenn 
\\[0.2cm]
\hspace*{1.3cm}
$a^2 + b^2 = c^2$ \quad und \quad $a < b$
\\[0.2cm]
gilt.  Beispielsweise ist $[3,4,5]$ ein geordnetes pythagoreisches Tripel, denn $3^2 + 4^2 = 5^2$.
\begin{enumerate}
\item Implementieren Sie eine Prozedur \texttt{pythagoras}, so dass der Aufruf
      \\[0.2cm]
      \hspace*{1.3cm}
      $\texttt{pythagoras}(n)$
      \\[0.2cm]
      die Menge aller geordneten  pythagoreischen Tripel $[a,b,c]$ berechnet, f�r die $c \leq n$ ist.
\item Ein pythagoreisches Tripel $[a,b,c]$ ist ein \blue{reduziertes} Tripel, wenn
      die Zahlen $a$, $b$ und $c$ keinen nicht-trivialen gemeinsamen Teiler haben.
      Implementieren Sie eine Funktion \texttt{isReduced}, die als Argumente drei nat�rliche Zahlen 
      $a$, $b$ und $c$ erh�lt und die genau dann \texttt{True} als Ergebnis zur�ck liefert,
      wenn das Tripel $[a, b, c]$ reduziert ist.
\item Implementieren Sie eine Prozedur \texttt{reducedPythagoras}, so dass der Aufruf
      \\[0.2cm]
      \hspace*{1.3cm}
      $\texttt{reducedPythagoras}(n)$
      \\[0.2cm]
      die Menge aller geordneten pythagoreischen Tripel $[a,b,c]$ berechnet, die reduziert sind.

      Berechnen Sie mit dieser Prozedur alle reduzierten geordneten pythagoreischen Tripel
      $[a,b,c]$, f�r die $c \leq 50$ ist. 
\end{enumerate}

\exercise
Nehmen Sie an, ein Spieler hat im Poker (Texas Hold'em) die beiden
Karten $\pair(8,\red{\heart})$ und $\pair(9,\red{\heart})$ erhalten.  Schreiben Sie ein
\textsc{SetlX}-Programm, dass die folgenden Fragen beantworten.
\begin{enumerate}
\item Wie gro� ist die Wahrscheinlichkeit, dass im Flop wenigsten zwei weitere Karten
      der Farbe \red{$\heart$} liegen?
\item Wie gro� ist die Wahrscheinlichkeit, dass alle drei Karten im Flop
      die Farbe \red{$\heart$} haben?
\end{enumerate}

\exercise
Ein \href{https://de.wikipedia.org/wiki/Anagramm}{Anagramm} eines gegebenen Wortes $v$ ist ein Wort $w$, dass
aus dem Wort $v$ durch Umstellung von Buchstaben entsteht.  Beispielsweise ist das Wort ``\texttt{atlas}'' ein
Anagramm des Wortes ``\texttt{salat}''.  Implementieren Sie eine Funktion $\texttt{anagram}(s)$, die f�r ein
gegebenes Wort $s$ alle W�rter berechnet, die sich aus dem Wort $s$ durch Umstellung von Buchstaben
ergeben.  Die Menge dieser W�rter soll dann als Ergebnis zur�ck gegeben werden.  Es ist nicht gefordert, dass
die Anagramme sinnvolle W�rter der deutschen Sprache sind.  Beispielsweise ist auch das Wort ``\texttt{talas}''
ein Anagramm des Wortes ``\texttt{salat}''.

\exercise
Nehmen Sie an, dass Sie $n$ W�rfel haben, deren Seiten mit den Zahlen 1 bis 6 bedruckt sind.  Weiter ist eine
feste Zahl $s$ vorgegeben.  Entwickeln Sie eine \textsc{SetlX}-Prozedur \texttt{numberDiceRolls}, so dass der Aufruf
\\[0.2cm]
\hspace*{1.3cm}
$\texttt{numberDiceRolls}(n, s)$ 
\\[0.2cm]
die Anzahl der M�glichkeiten berechnet, mit $n$ W�rfeln in der Summe die Zahl $s$ zu w�rfeln.  Beispielsweise
soll $\texttt{numberDiceRolls}(3, 5)$ den Wert 6 liefern, denn es gibt 6 M�glichkeiten, um mit drei W�rfeln in
der Summe eine 5 zu w�rfeln:
\\[0.2cm]
\hspace*{1.3cm}
$\langle1, 1, 3\rangle, \langle1, 2, 2\rangle, \langle1, 3, 1\rangle, \langle2, 1, 2\rangle, \langle2, 2,
1\rangle, \langle3, 1, 1\rangle$

\noindent
\textbf{Hinweis}:  Implementieren Sie eine geeignete Hilfsfunktion.
\end{document}

%%% Local Variables: 
%%% mode: latex
%%% TeX-master: t
%%% ispell-local-dictionary: "deutsch8"
%%% End: 
